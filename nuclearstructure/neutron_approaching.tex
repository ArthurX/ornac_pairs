\section{A neutron approaching a proton}


How does the approach of a neutron to a proton look like? And why are there oscillations in the neutron absorption ratio depending on the speed of the neutron? Figure shows the graphic.
Let's assume that the proton is stationary and that the neutron approaches. It would be easiest, if the sub-c magnetic field of both would align and so help guide the coupling, but they are opposing. This opposition of the magnetic fields is also quite necessary, because the magnetic field of deuterium is much smaller than that of a proton or a neutron apart, so adding them would go in the wrong direction, subtracting is better. If you look at figure of the end state, you see a deuterium stick, with all ornac-pairs rotating in the same direction. Three of them are positive and three negative, together the magnetic fields almost cancel.

So the neutron is approaching, while rotating. The rotational speed depends on the linear speed. So the sub-c magnetic fields alternate between attractive and repulsive for each rotation. The hyper-c magnetic fields can only couple during a short part of the rotation. If the middle ornac-pairs couple, due to the freedom of the wiggle angle, the force is repulsive, otherwise attractive. So we have a neutron approaching a proton and there are certain speeds at which there is a greater chance that they couple, due to these interactions.

\paragraph{A neutron approaching a 3-Helium stick:}
3-He has a higher chance of absorbing a neutron than the proton as above. In the proton the middle ornac-pair can wiggle and so cause a repulsive force. In 3-He the ornac stick is longer and consists of 9 pairs: 5 positive pairs and 4 negative ones. So it are actually two sticks each coupled by their own hyper-c field, and kept together by the Coulomb force, so there is no way that the axis can wiggle. If the neutron passes by it can only couple attractive. There is no chance for a repulsive coupling and so the absorption ratio is greater.

\paragraph{A neutron approaching a deuterium stick:}
If we compare a deuterium stick with 3-He we notice that the neutron absorption for deuterium is low. So we can continue the reasoning from the 3-Helium stick, because deuterium consists of six ornac-pairs: 3 positive, 3 negative, so two sticks are formed, hold together by Coulomb force, so there is no wiggle angle and one would expect the neutron absorption to be higher than a proton, except that it had a single side to couple. It has a lower chance to couple, so we have a proton, if a neutron couples to its top side, it becomes deuterium, but then it is very difficult for another neutron to couple to its bottom side. So we can conclude that while the magnetic field helps to couple the neutron to the top side of a proton, it inhibits coupling to the bottom side.


