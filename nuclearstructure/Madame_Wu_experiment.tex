\section{The Madame Wu %吴 
experiment}
\[^{60}_{27}Co \Rightarrow {^{60}_{28}Ni} + e + \nu + 2 \gamma\]

\paragraph{}
Two  gamma ray photons can mean two spin flips
% 吴健雄
Wú Jiànxíong
% https:\\en.m.wikipedia.org\wiki\Wu_experiment
% http:\\bnrc.berkeley.edu\Famous-Women-in-Physical-Sciences-and-Engineering\chien-shiung-wu.html


In 1956 Wú Jiànxíong %吴健雄 
was inspired by her friends, theoretical physicists Tsung-Dao Lee and Chen-Ning Yang, to do an experiment to see, whether the $\beta$ decay was sensitive to the direction of the magnetic field. She chose to use decay of Cobalt 60 to Nickel 60, where by one electron and two $\gamma$-rays are emitted. The neutrino is ignored.


The experiment has to be performed at low temperatures, otherwise the thermal motion disturbs the orientation of the spins of the nuclea.
So a small layer of Cobalt 60 is coated onto a substrate and is cooled to 1.2 Kelvin.
To reach the low temperatures needed, adiabatic demagnetization cooling is used to reach 0.003 Kelvin.


After the magnetic field is switched off, a small field is switched on to orient the spins of the atoms.
Then the directions of  the emitted electrons and the $\gamma$ photons is measured.
And indeed the electrons come out in a preferential direction. If the magnetic field points up the electrons go down and vice versa. Not all electrons but 60 percent.


So let's look how that works out in our picture.
Firstly we know that the nucleus aligns along an ornac-stick with an odd number of nuclides, so let's assume three here.


Secondly if a $\beta$ decay happens, it is an ornac-stick of the n-p-n type, which transforms to p-n-p.
So let's align our n-p-n ornac-stick to the magnetic field and let the electron escape.
Figure sketches the idea.


The magnetic field points up, we only draw one ornac-stick and ignore the other sticks, the electron will go down. The ornac stick has at each end an ornac-pair of which the positive $\sfrac{1}{3}$ ornac is stationary and visible, while the negative ornac rotates, which creates the magnetic and hyper magnetic fields. So according to the right hand rule this whole axis is turning around the left hand fist.


So let's assume the decay reaction starts at the bottom. So the bottom ornac-pair, with the positive $\sfrac{1}{3}$ ornac stationary decays. While we don't know the exact scenario, we start with an hypothetical one and later take a step back to evaluate how much sense it makes and possible improve it.


Here we go:


The bottom ornac-pair vibrates along the axis. It moves too much to the bottom and shoots up. It comes to close into the influence of the opposite charged ornac-pair above it.
It decays into a lower energy $\sfrac{2}{3}$ ornac-pair, with a wider orbit. The difference in energy and angular momentum is released in an electron and neutrino of opposing angular momentum. The neutrino and electron are loosely coupled and move for a short distance together. The short lived constellation of electron and neutrino is usually called a W-boson. The electron-neutrino constellation moves downward. That is what is measured in the Madame Wu experiment.


 This leaves some linear momentum for the newly formed $\sfrac{+2}{3}$ ornac-pair in the upward direction.
So the newly formed $\sfrac{+2}{3}$ ornac-pair moves up. There is an attractive hyper-c force between equal particles. It propagate up the ornac-stick, until it reaches the top and becomes the new top of the stick.
In the process the direction of ordinary magnetic field of the stick is reversed.


The electron-neutrino constellation moves further downward and falls apart leaving the trajectory of the electron in about the same direction.


So question in the end is, is the electron moving because of the magnetic field and its spin or is this due to the fact, that the magnetic field puts the atomic nucleus in a special orientation with regards to its magnetic ornac-axis and is decay taking place along that axis?


A further experiment could use Lithium 9, which decays to Beryllium 9. Lithium 9 is supposed to have three axis with each a n-p-n stick. This severely limits the number of directions in which an electron could be emitted. Or maybe a little bit more difficult, Tritium could be used, having only one axis even more strongly limited in its direction.


Goldhaber experiment
% https:\\de.m.wikipedia.org\wiki\Goldhaber-Experiment
% https:\\www.google.de\search?q=goldhaber+experiment&ie=utf-8&oe=utf-8&gws_rd=cr&ei=HBwSWMuYBMTgacXFsbAJ



