\chapter{Feynman and the double slit}
\section{The electron double slit experiment}
In the 1950-ties Richard Feynman proposed a double slit experiment with electrons to educate about the quantum nature of particles. The idea is as follows: to show that interference of the electrons only happens, if they have the possibility to travel through both slits at the same time, an additional barrier is added. This barrier has a square hole which is bigger than both slits and it can move from left to right, so that only the left slit is open, both slits are open or only the right slit is open. Interference is only expected if both slits are open.

\paragraph{}
Sixty years later Roger Bach et al 2013 New J. Phys. 15 033018
 http://dx.doi.org/10.1088/1367-2630/15/3/033018  
performed this experiment and as expected full interference shows up, if both slits are open. Figure sketches the setup and the result. More interesting is this image: the barrier is covering the right slit and leaves the left one open. An partial interference pattern is discernable: at the left side the electrons are deflected with a periodicity, at the right side it is more a continuous blob. What is happening? The electrons are only traveling through the left slit.
We know what is happening, we have a hunch. The electrons are coming in with a fixed speed. When they pass through the slit, they experience a push to the left. It can be one, two or ten pushes, but it is always a multiple of the one unit push. What is creating this push? And why has it a fixed impulse? We have to start with a simpler situation. Figure sketches the idea: all the matter of the material between the two slits is concentrated in a very narrow line in the middle and the electrons only move exactly through the middle of the left slit. The electrons couple to the nuclea of the middle. The electrons rotate while moving forward. At the moment the axis of the electron is perpendicular to its direction of motion, it points in the direction of material of the middle. Its hyper-c field is narrow and if in the middle material one of the nuclea is at the right spot and one of its ornac-axis is completely aligned with the hyper-c field of the electron, then those two hyper-c fields will couple and create the unit push. As the electron makes a number of rotations, while moving past the concentrated material to the left, there are many chances to couple to a nucleus and so it can experience one or more pushes, but always a multiple of the same unit push.
The push has a fixed size, because everything is fixed: the speed of the electron, the distance between the electron and the nucleus and hence the size of the force and, because of the DeBroglie wavelength,the duration of the force. And the impulse is the integral of force over time.
The force between the electron and the nuclea is always repulsive, so the interference pattern only shows up on the left side. And the blob on the right side is due to continuous material to the left of the slit, which produces pushes of varying impulse, because the distance between the electron and the nucleus can have any size. 
The image as our simplified situation would create, is sketched in figure. Instead of the sinusoidal density fluctuations of the real world experiment, we get a number of sharp defined lines to the left and the blob at the right side. To recreate the real world experiment we can slowly broaden the range, in which the electrons move and broaden the concentration of the material. In that way we bring variation in the distance between the electron and the nuclea and hence the force will vary and our image will get blurred.


\paragraph{}
We have to add a detail to our image. If the beams of hyper-c force cross between the electron and the nucleus, it depends on the duration how large the change in impulse will be. So if the electron rotates clock wise, there will be a point to the right along its beam, where there is a temporary zero speed, because the rotational speed and the forward speed cancel. Let's call that the dead point. This dead point allows some time, so that the visual distinct unit push can show up. Because the electron not only couples to that one or five nuclea with a fixed unit push, it also couples to many other nuclea, but for a much shorter time, so they don't show up with a distinctive size. Let's try to calculate the distance of the dead point.
The formula for the dead point radius is a combination of the one for the dead point for a rotating thing.

\[2\pi f r_{dead} = v\]

This formula relates the ratio $\frac{f}{v}$ to $r_{dead}$. The second formula relates $f$ and $v$  to each other:

The formula for the wavelength is:

The sharp lines of our simplified situation do appear in the real world, if the matter of the nucleus is confined to well defined places. In crystal-spectrography electrons or x-ray photons are shot through the crystals and the deviation from their path is used to measure the distances between the nuclea. The radius of the nucleus is in femtometer scale $10^{-15} m$ , while the atom has a radius in the angstrom scale $10^{-10} m$, so the point to which the electron couples is defined till $10^{-5}$. On the other hand the position of the path of the electron is also constrained, if it moves too close to the nucleus, the chance of being randomly deflected by a bound electron increases enormously.

Let us try to get some more specific numbers. The electron has a wavelength of 60 nm, so it makes 16 rotations for every $\mu m$ it moves forward. The metal, from which the double slit structure is made, is $50 \mu m$ thick, so every electron has 800 chances to couple to an ornac-axis of a nucleus. If we again look at the image of the experiment of a and b,

\paragraph{}
The formula, which describes the distances between maxima and the central axis, is well known and is as follows:

  \[\sin{\theta_n}= n \frac{\lambda }{d} \]
in which $\theta$ is the angle between the original trajectory of the electron and its new deviated path, $n$ is any integer, which describes the number of the maximum, $\lambda$ is the wavelength and $d$ is distance between the two slits. In the formula the distance between the two slits $d$ is used and we know the formula is right. But it is not a fundamental difference if we use half this distance $d$, namely the distance between the middle of the slit and the middle of the of the material in the middle.

\paragraph{}
If the deviation angle $\theta$ is small, $\sin{\theta} $  can be replaced by $\frac{y_n}{L}$, where $y_n$ is the lateral displacement and $L$ the distance between the slit and the projection screen:
   \[y_n = n \frac{\lambda L }{d} \]
Or the angle by which the course of the electron is deviated for each individual impulse is:
  \[ \theta = \frac{\lambda }{d} \]
So we can calculate the individual impulse and try to make an educated guess about is duration and find the average size of the force pushing the electron away.
(in which$n$ is any integer, $\lambda$ is the wavelength, $L$ is distance between slit and screen and $d$ is distance between the two slits.)

\paragraph{}
What will happen to the angle of deflection, if we double the distance between the middle of the slit and the concentrated material? The distance between the electron and the nucleus doubles, so the force between them, during the moment they couple, becomes four times as weak. But because the distance is twice as large, the duration of the coupling is twice as long. So the impulse is twice as weak. Figure sketches the idea: two beams cross each other and if the distance between them doubles, the duration of the crossing doubles. It also shows the change in the shape of the impulse. Of course this agrees well with the formula for the bending of light and matter in the double slit and the grating.
Now, what will happen, if we double the speed of the electron? Everything stays the same as in the simplified situation, except the speed. If the speed of the electron doubles, its momentum doubles, 

\paragraph{}
(the impulse needs to quadruple to keep the same angle of deflection, because speed v appears squared in F= m v2 / r.)
If the speed of the electron doubles, its wavelength halves, so the angular speed, with which it rotates, quadruples. This means that the dead point distance moves two times closer to the electron, hence the force between the ornac pairs in the electron and the nucleus quadruples.
\section{Photons in the slit}
The same experiment was first done with photons. So let's try that. The photons move through the middle of the slit. They are polarized. In this way they only can couple perpendicular to the material beside the slit. The photon consists of an ornac-pair, two opposing charges rotating around a central middle point. The middle point moves with the speed of light. So one of the charges moves above c in a curvatory trajectory and this creates both the ordinary magnetic field of the photon and the hyper-c field of the photon. This hyper-c field of the photon can couple to the hyper-c field of the quark ornac-pairs in the protons and neutrons of the nuclides besides the slit.
So the same mechanism, which gives the electrons a unit kick, also gives the photons a unit kick. The electrons can only experience a repulsive force. but the photons can also be attracted, depending on whether a positive or negative ornac is above c. Also as the coupling is not really in the center, the photon will be a little twisted around its axis of motion. So the direction of its polarization will change a little. Of course the photon also has numerous chances to couple and hence can experience a number of pushes or pulls.
Doubling the distance works in the same way as with the electron. So no need to go into that again. Doubling the energy or frequency of the photons, makes the ornac-pair rotate faster,  creating a larger magnetic and hyper-c field, this larger hyper-c field will couple the photon stronger, so a higher frequency results in stronger bending.
At this moment we can try to deduct, how the coupling between the ornac-pair of the photon and the ornac-pairs of the nuclea depend or alternatively how the distance between the photon ornac-pairs vary with frequency.

