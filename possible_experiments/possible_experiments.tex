\chapter{Possible experiments}

The structure of 3 He and 4 He should be sufficiently different to be detected in experiments. As measured by the charge radius is respectively  and. 3 He consists of a single stick of 9 ornac-pairs and it has a large magnetic moment. if 3 He is placed in a strong magnetic field and all the nuclides are aligned with the field, one would expect to get a different result based on the orientation. The electrons which are fired on the sample will be deflected. Even the polarization of the electrons plays a role.

To execute the  measurement one first switches on the magnetic field, waits till all the 3He nuclides are oriented with the field, then one switches off the magnetic field, fires the electrons to probe for the charge radius.

Another similar experiment could involve the absorption of neutrons by 3 He. The same magnetic field would influence the absorption depending on the polarization of the neutrons.

