

%
% Helium3_Absorbs_Neutron
%
%\begin{figure}
%\begin{tikzpicture}[scale=0.3, rotate=0]






% The approaching neutron
\begin{scope}[scale=1, rotate=0 ,shift={(0,-22)} ]
\begin{scope}[scale=1, rotate=-30 ,shift={(0,0)} ]




% Rotation arrow
\draw[black, dashed, very thick, ->] (8,6)  arc[radius = 5.6, start angle= 45, end angle= -45];



% Ornac axis
\draw[green, dashed, very thick, <-] (4,6) -- (4,-2) ;





\path 
(-4,2) node   {p};




% Up quark ornac-pair
\path 
(-2,4) node  (red) {$+\frac{2}{3}$};

\filldraw[fill=blue, draw=black] (4,4) circle [radius=0.5];

\filldraw[fill=black, draw=black ,rounded corners=1pt] (1, 3.9) rectangle (7, 4.1);



% Down quark ornac-pair
\path 
(-2,2) node  (red) {$-\frac{1}{3}$};

\filldraw[fill=red, draw=black] (4,2) circle [radius=0.5];

\filldraw[fill=black, draw=black ,rounded corners=1pt] (0, 1.9) rectangle (8, 2.1);


% Up quark ornac-pair
\path 
(-2,0) node  (red) {$+\frac{2}{3}$};

\filldraw[fill=blue, draw=black] (4,0) circle [radius=0.5];

\filldraw[fill=black, draw=black ,rounded corners=1pt] (1,-0.1) rectangle (7, 0.1);





\end{scope}
\end{scope}


















%
% Tritium: quark ornac-pairs with ornac axis
%
%\begin{figure}
%\begin{tikzpicture}[scale=0.4, rotate=0]



% Ornac axis
\draw[green, dashed, very thick, <-] (4,4) -- (4,-18) ;


%
% Neutron top
%
\path 
(-4,0) node   {n};


% Down quark ornac-pair
\path 
(-2,2) node  (red) {$-\frac{1}{3}$};

\filldraw[fill=red, draw=black] (4,2) circle [radius=0.5];

\filldraw[fill=black, draw=black ,rounded corners=1pt] (0, 1.9) rectangle (8, 2.1);


% Up quark ornac-pair
\path 
(-2,0) node  (red) {$+\frac{2}{3}$};

\filldraw[fill=blue, draw=black] (4,0) circle [radius=0.5];

\filldraw[fill=black, draw=black ,rounded corners=1pt] (1,-0.1) rectangle (7, 0.1);



% Down quark ornac-pair
\path 
(-2,-2) node  (red) {$-\frac{1}{3}$};

\filldraw[fill=red, draw=black] (4,-2) circle [radius=0.5];

\filldraw[fill=black, draw=black ,rounded corners=1pt] (0, -1.9) rectangle (8, -2.1);


%
% Proton midle
%

\path 
(-4,-7) node   {p};

% Up quark ornac-pair
\path 
(-2,-5) node  (red) {$+\frac{2}{3}$};

\filldraw[fill=blue, draw=black] (4,-5) circle [radius=0.5];

\filldraw[fill=black, draw=black ,rounded corners=1pt] (1, -4.9) rectangle (7, -5.1);



% Down quark ornac-pair
\path 
(-2,-7) node  (red) {$-\frac{1}{3}$};

\filldraw[fill=red, draw=black] (4,-7) circle [radius=0.5];

\filldraw[fill=black, draw=black ,rounded corners=1pt] (0, -6.9) rectangle (8, -7.1);


% Up quark ornac-pair
\path 
(-2, -9) node  (red) {$+\frac{2}{3}$};

\filldraw[fill=blue, draw=black] (4,-9) circle [radius=0.5];

\filldraw[fill=black, draw=black ,rounded corners=1pt] (1,-8.9) rectangle (7, -9.1);



%
% Neutron lower
%
\path 
(-4,-14) node   {n};


% Down quark ornac-pair
\path 
(-2,-12) node  (red) {$-\frac{1}{3}$};

\filldraw[fill=red, draw=black] (4,-12) circle [radius=0.5];

\filldraw[fill=black, draw=black ,rounded corners=1pt] (0, -11.9) rectangle (8, -12.1);


% Up quark ornac-pair
\path 
(-2,-14) node  (red) {$+\frac{2}{3}$};

\filldraw[fill=blue, draw=black] (4,-14) circle [radius=0.5];

\filldraw[fill=black, draw=black ,rounded corners=1pt] (1,-13.9) rectangle (7, -14.1);



% Down quark ornac-pair
\path 
(-2,-16) node  (red) {$-\frac{1}{3}$};

\filldraw[fill=red, draw=black] (4,-16) circle [radius=0.5];

\filldraw[fill=black, draw=black ,rounded corners=1pt] (0, -15.9) rectangle (8, -16.1);









%\end{tikzpicture}
%\caption{Hetrium Tritium 4 releases proton. The Tritium stays stationary as the proton is kicked out.
%\label{Fig:Helium3_becomes_Tritium}}
%\end{figure}

% End Helium3


