

%
% Helium3_Absorbs_Neutron
%
%\begin{figure}
%\begin{tikzpicture}[scale=0.3, rotate=0]




% The approaching neutron
\begin{scope}[scale=1, rotate=0 ,shift={(0,16)} ]
\begin{scope}[scale=1, rotate=30 ,shift={(0,0)} ]


% Ornac axis
\draw[green, dashed, very thick, <-] (4,4) -- (4,-4) ;
%\path (4,-4) node [below]  {ornac axis};

% Rotation arrow
\draw[black, dashed, very thick, ->] (8,4)  arc[radius = 5.6, start angle= 45, end angle= -45];



\path 
(-4,0) node   {n};

% Down quark ornac-pair
\path 
(-2,2) node  (red) {$-\frac{1}{3}$};

\filldraw[fill=red, draw=black] (4,2) circle [radius=0.5];

\filldraw[fill=black, draw=black ,rounded corners=1pt] (0, 1.9) rectangle (8, 2.1);


% Up quark ornac-pair
\path 
(-2,0) node  (red) {$+\frac{2}{3}$};

\filldraw[fill=blue, draw=black] (4,0) circle [radius=0.5];

\filldraw[fill=black, draw=black ,rounded corners=1pt] (1,-0.1) rectangle (7, 0.1);



% Down quark ornac-pair
\path 
(-2,-2) node  (red) {$-\frac{1}{3}$};

\filldraw[fill=red, draw=black] (4,-2) circle [radius=0.5];

\filldraw[fill=black, draw=black ,rounded corners=1pt] (0, -1.9) rectangle (8, -2.1);





\end{scope}
\end{scope}





















% Helium3 stationary
% Ornac axis
\draw[green, dashed, very thick, <-] (4,11) -- (4,-11) ;
\path 
(4,-11) node [below]  {ornac axis};

%
% Proton Top
%

\path 
(-4,7) node   {p};

% Up quark ornac-pair
\path 
(-2,9) node  (red) {$+\frac{2}{3}$};

\filldraw[fill=blue, draw=black] (4,9) circle [radius=0.5];

\filldraw[fill=black, draw=black ,rounded corners=1pt] (1, 8.9) rectangle (7, 9.1);



% Down quark ornac-pair
\path 
(-2,7) node  (red) {$-\frac{1}{3}$};

\filldraw[fill=red, draw=black] (4,7) circle [radius=0.5];

\filldraw[fill=black, draw=black ,rounded corners=1pt] (0, 6.9) rectangle (8, 7.1);


% Up quark ornac-pair
\path 
(-2, 5) node  (red) {$+\frac{2}{3}$};

\filldraw[fill=blue, draw=black] (4,5) circle [radius=0.5];

\filldraw[fill=black, draw=black ,rounded corners=1pt] (1,4.9) rectangle (7, 5.1);








%
% Neutron middle
%
\path 
(-4,0) node   {n};


% Down quark ornac-pair
\path 
(-2,2) node  (red) {$-\frac{1}{3}$};

\filldraw[fill=red, draw=black] (4,2) circle [radius=0.5];

\filldraw[fill=black, draw=black ,rounded corners=1pt] (0, 1.9) rectangle (8, 2.1);


% Up quark ornac-pair
\path 
(-2,0) node  (red) {$+\frac{2}{3}$};

\filldraw[fill=blue, draw=black] (4,0) circle [radius=0.5];

\filldraw[fill=black, draw=black ,rounded corners=1pt] (1,-0.1) rectangle (7, 0.1);



% Down quark ornac-pair
\path 
(-2,-2) node  (red) {$-\frac{1}{3}$};

\filldraw[fill=red, draw=black] (4,-2) circle [radius=0.5];

\filldraw[fill=black, draw=black ,rounded corners=1pt] (0, -1.9) rectangle (8, -2.1);


%
% Proton bottom
%

\path 
(-4,-7) node   {p};

% Up quark ornac-pair
\path 
(-2,-5) node  (red) {$+\frac{2}{3}$};

\filldraw[fill=blue, draw=black] (4,-5) circle [radius=0.5];

\filldraw[fill=black, draw=black ,rounded corners=1pt] (1, -4.9) rectangle (7, -5.1);



% Down quark ornac-pair
\path 
(-2,-7) node  (red) {$-\frac{1}{3}$};

\filldraw[fill=red, draw=black] (4,-7) circle [radius=0.5];

\filldraw[fill=black, draw=black ,rounded corners=1pt] (0, -6.9) rectangle (8, -7.1);


% Up quark ornac-pair
\path 
(-2, -9) node  (red) {$+\frac{2}{3}$};

\filldraw[fill=blue, draw=black] (4,-9) circle [radius=0.5];

\filldraw[fill=black, draw=black ,rounded corners=1pt] (1,-8.9) rectangle (7, -9.1);







%\end{tikzpicture}
%\caption{Helium3 absorbs neutron. The Helium3 is stationary ad the neutron approaches.
%\label{Fig:Helium3_Absorbs_Neutron}}
%\end{figure}

% End Helium3


