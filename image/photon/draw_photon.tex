
\section{Photon vertical with speed vectors}
\begin{tikzpicture}[scale=1, rotate=0]

% photon vertical with speed vectors

\path 
(0,0) node[circle,draw, fill=red]  (red) {$-\frac{1}{3}$}
(0,6) node[circle,draw, fill=blue] (blue) {$+\frac{1}{3}$};
\draw[black] (red) -- (blue)         node[pos=0.5](center){};
\filldraw
 (center) circle (2pt);
\draw[->, black] (center) -- ++(1,0) node [anchor=west]{speed $c$};
\draw[->, black] (blue) -- ++(2,0) node [anchor=west]{speed $2c$};
\draw[->, dotted] (red) -- ++(0.7,0) node [anchor=west]{speed $\emptyset$};

\draw[->,dashed] (-5,4) to[out=60,in=180] (blue);

\end{tikzpicture}


\section{Photon vertical with green and yellow area for \\
 sub and hyper c  indication speed}
\begin{tikzpicture}[scale=1, rotate=0]

% photon vertical with green and yellow area for 
% sub and hyper c  indication speed 

\filldraw[fill=yellow, draw=black] (0,0) -- (60:6) 
node[pos=0.75, ](yellow) {} 
arc (60:120:6) -- (0,0);
\draw[->, dashed, black] (yellow) -- ++(2,0) node [right, align=left]
{Yellow area with speed above $c$: \\ 
a special case of the magnetic field: \\
only visible to other particles,  \\ 
that rotate around the same axis, \\
like another photon \\
};

\filldraw[fill=green, draw=black] (0,0) -- (60:3) 
node[pos=0.6, ](green) {} 
arc (60:120:3) -- (0,0);
\draw[->, dashed, black] (green) -- ++(3.4,0) node [right , align=left]
{
Green area with speed below $c$: \\ 
the ordinary magnetic field: \\ 
visible to all};


\path 
(0,0) node[circle,draw, fill=red]  (red) {$-\frac{1}{3}$}
(0,6) node[circle,draw, fill=blue] (blue) {$+\frac{1}{3}$};
\draw[black] (red) -- (blue)         node[pos=0.5](center){};
\filldraw
 (center) circle (2pt);
\draw[->, black] (center) -- ++(1,0) node [anchor=west]{speed $c$};
\draw[->, black] (blue) -- ++(2,0) node [anchor=west]{speed $2c$};
\draw[->, dotted] (red) -- ++(0.7,0) node [anchor=west]{speed $\emptyset$};

%\draw[->,dashed] (-5,4) to[out=60,in=180] (blue);



\end{tikzpicture}



\section{Photon horizontal with speed vectors}
\begin{tikzpicture}[scale=1, rotate=0]


% photon horizontal with speed vectors



\path 
(0,0) node[circle,draw, fill=red](red) {$-\frac{1}{3}$}
(6,0) node[circle,draw, fill=blue](blue) {$+\frac{1}{3}$};
\draw[black] (red) -- (blue)          node[pos=0.5](center){};
\filldraw
 (center) circle (2pt);
%\draw[->, black] (center) -- ++(1,0) node [anchor=west]{};

%% Arrows indicating the speed components from blue
\draw[->, black, dashed] (blue) -- node[above] { $1c$} ++(2,0) node [](west){};
\draw[->, black, dashed] (blue) -- node[left] { $1c$} ++(0,-2) node [](south){};
\draw[ black, dotted] (west) -- ++(0,-2) node [](southwest){};
\draw[ black, dotted] (south) -- (southwest) node []{};
\draw[->, black] (blue) --  (southwest) node[below, right] { $1.4c$};

%% Arrows indicating the speed components from red
\draw[->, black, dashed] (red) -- node[above] { $1c$} ++(2,0) node [](redwest){};
\draw[->, black, dashed] (red) -- node[left] { $1c$} ++(0,2) node [](redsouth){};
\draw[ black, dotted] (redwest) -- ++(0,2) node [](redsouthwest){};
\draw[ black, dotted] (redsouth) -- (redsouthwest) node []{};
\draw[->, black] (red) --  (redsouthwest) node[below, right] { $1.4c$};




\end{tikzpicture}

\section{Photon diagonal with speed vectors}

\begin{tikzpicture}[scale=1, rotate=0]


% photon diagonal with speed vectors



\path 
(0,4) node[circle,draw, fill=red](red) {$-\frac{1}{3}$}
(4,0) node[circle,draw, fill=blue](blue) {$+\frac{1}{3}$};
\draw[black] (red) -- (blue)          node[pos=0.5](center){};
\filldraw
 (center) circle (2pt);
\draw[->,dashed, black] (center) -- node[above] { $1c$} ++(2,0) node [anchor=west]{};

%% Arrows indicating the speed components from blue
\draw[->, black, dashed] (blue) -- node[above] { $1c$} ++(2,0) node [](west){};
\draw[->, black, dashed] (blue) -- node[left] { $1c$} ++(225:2) node [](south){};
\draw[ black, dotted] (west) -- ++(225:2) node [](southwest){};
\draw[ black, dotted] (south) -- (southwest) node []{};
\draw[->, black] (blue) --  (southwest) node[below, right] { $0.76c$};

%% Arrows indicating the speed components from red
\draw[->, black, dashed] (red) -- node[above] { $1c$} ++(2,0) node [](redwest){};
\draw[->, black, dashed] (red) -- node[left] { $1c$} ++(45:2) node [](redsouth){};
\draw[ black, dotted] (redwest) -- ++(45:2) node [](redsouthwest){};
\draw[ black, dotted] (redsouth) -- (redsouthwest) node []{};
\draw[->, black] (red) --  (redsouthwest) node[below, right] { $1.84c$};

\end{tikzpicture}


\section{Photon going through cycloid cycle in colors}
\begin{tikzpicture}[scale=1, rotate=0]

% photon going through cycloid cycle in colors
 \foreach \x in {0, 0.05,...,1}
  {
  \definecolor{reddish}{hsb}{\x, 1, 1}
  \draw[draw=none, fill=reddish] 
   ({((2*pi*\x + sin(2*pi*\x r)) },{1 + cos(2*pi*\x r) }) circle [radius=0.5];

  }

 \foreach \x in {0, 0.05,...,1}
  {
  \definecolor{reddish}{hsb}{\x, 1, 1}
  \draw[draw=none, fill=reddish] 
   ({((2*pi + 2*pi*\x + sin(2*pi*\x r)) },{1 + cos(2*pi*\x r) }) circle [radius=0.5];
  }

\path
(2*pi,2) node (topdoublespeed) {}
(3*pi,0) node (deadbottom2) {};
\draw[<-, thick] (topdoublespeed) +(0.5, 0.5) -- ++(2,2) node (topdoublespeedtext){}; 
\draw[<-, thick] (deadbottom2)+(0.4,-0.4) -- ++(2,-2) node (deadbottomtext) {}; 
\path
(topdoublespeedtext) node [anchor=south west,  align=right]
{\textbf{TopDoubleSpeed} \\
The speed is max. \\ 
The speed is double the average. \\
Red indicates high speed.};
\path
(deadbottomtext) node [below , align=right]
{\textbf{DeadBottom} \\
The direction of movement is reversed. \\  
There is a slight moment of zero speed. \\
Turquoise blue indicates the low speed.};


\end{tikzpicture}



\section{Cycloid cycle in colors single ornac of a photon \\ with wheel}
\begin{tikzpicture}[scale=1, rotate=0]

% single ornac of a photon going through cycloid cycle in colors

 \foreach \x in {0, 0.05,...,1}
  {
  \definecolor{reddish}{hsb}{\x, 1, 1}
  \draw[black, fill=reddish] 
   ({((2*pi*\x + sin(2*pi*\x r)) },{1 + cos(2*pi*\x r) }) circle [radius=0.09];
  }
% second cycle
 \foreach \x in {0, 0.05,...,1}
  {
  \definecolor{reddish}{hsb}{\x, 1, 1}
  \draw[black, fill=reddish] 
   ({((2*pi + 2*pi*\x + sin(2*pi*\x r)) },{1 + cos(2*pi*\x r) }) circle [radius=0.09];
  }


% wheel
\path
(0,1) node (hub) {};
\draw[black, thick]
(hub) circle[radius=1];
% arrow

%\draw[->, thick]
%(hub)++(215:1.5) arc  (315:225:1.5);
% spokes
 \foreach \phi in {0, 0.1,...,1}
  {
%  \definecolor{reddish}{hsb}{\x, 1, 1}
  \draw[black, thick] 
   (hub)-- ++(360*\phi :1);
  }

\end{tikzpicture}



\section{Single ornac of a photon  going through two cycloid cycle in red}
\begin{tikzpicture}[scale=1, rotate=0]

% single ornac of a photon  going through two cycloid cycle in red

%% red ornac
 \foreach \x in {0, 0.05,...,2}
  {
  
  \draw[black, fill=red] 
   ({((2*pi*\x + sin(2*pi*\x r)) },{1 + cos(2*pi*\x r) }) circle [radius=0.05];

  }

\path
(3*pi,0) node (bottom){};

\draw[<-, thick] (bottom) -- ++(1,-1) node (slowspeed){}; 
\path
(slowspeed) node [anchor=west, align=right]{slowest speed = $\emptyset$};

\end{tikzpicture}




\section{Single ornac of a photon  going through two cycloid cycle in red
 with green oval to indicate the under c area}
\begin{tikzpicture}[scale=1, rotate=0]

% single ornac of a photon  going through two cycloid cycle in red
% with green oval to indicate the under c area

% greeen oval
\path
(pi,0) node (bottom1){}
(3*pi,0) node (bottom2){};
\draw [draw = none, fill = green]
 (bottom1) circle [ x radius = 0.4, y radius = 1 ]
 (bottom2) circle [ x radius = 0.4, y radius = 1 ]node (greenoval) {};

\draw[<-, thick] (greenoval)+(0.4,-0.4) -- ++(2,-2) node (slowspeed){}; 
\path
(slowspeed) node [below right , align=right]
{green oval \\ indicates area \\ with  speed bellow $c$};


%% red ornac
 \foreach \x in {0, 0.05,...,2}
  {
    \draw[black, fill=red] 
   ({((2*pi*\x + sin(2*pi*\x r)) },{1 + cos(2*pi*\x r) }) circle [radius=0.05];
  }


\end{tikzpicture}



\section{Two ornac of a photon  going through two cycloid cycle in red}
\begin{tikzpicture}[scale=1, rotate=0]

% two ornac of a photon  going through two cycloid cycle in red

%% red ornac
 \foreach \x in {0, 0.05,...,2}
  {
    \draw[black, fill=red] 
   ({((2*pi*\x + sin(2*pi*\x r)) },{1 + cos(2*pi*\x r) }) circle [radius=0.05];
  }


%% blue ornac
 \foreach \x in {0, 0.05,...,2}
  {
    \draw[black, fill=blue] 
   ({((2*pi*\x + sin((2*pi*\x -pi) r)) },{1 + cos((2*pi*\x -pi) r) }) circle [radius=0.05];
  }

\end{tikzpicture}


\section{Two ornac of a photon  going through two cycloid cycle in red and blue
 with green oval to indicate the under c area}
\begin{tikzpicture}[scale=1, rotate=0]

% two ornac of a photon  going through two cycloid cycle in red and blue
% with green oval to indicate the under c area

% greeen ovals
 \foreach \p in {1,...,3}
    \draw [draw = none, fill = green]
     (pi*\p,0) circle [ x radius = 0.4, y radius = 1 ];

\path
(3*pi,0) node (greenoval) {};
\draw[<-, thick] (greenoval)+(0.4,-0.4) -- ++(2,-2) node (slowspeed){}; 
\path
(slowspeed) node [below right , align=right]
{green oval \\ indicates area \\ with  speed bellow $c$};

\path
(1*pi,0) node (greenoval1) {}
(2*pi,0) node (greenoval2) {};
\draw[<-, thick] (greenoval1)+(0.4,-0.4) -- (1.5*pi,-2) node (alternatingly){}; 
\draw[<-, thick] (greenoval2)+(-0.4,-0.4) -- (1.5*pi,-2) node {}; 
\path
(alternatingly) node [below , align=center]
{positive and negative charges \\ come  \textbf{alternatingly} in the \\ window of perception};


%% red ornac
 \foreach \x in {0, 0.05,...,2}
  {
    \draw[black, fill=red] 
   ({((2*pi*\x + sin(2*pi*\x r)) },{1 + cos(2*pi*\x r) }) circle [radius=0.05];
  }


%% blue ornac
 \foreach \x in {0, 0.05,...,2}
  {
    \draw[black, fill=blue] 
   ({((2*pi*\x + sin((2*pi*\x -pi) r)) },{1 + cos((2*pi*\x -pi) r) }) circle [radius=0.05];
  }

\end{tikzpicture}
