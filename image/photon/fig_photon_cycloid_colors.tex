\begin{figure}\begin{tikzpicture}[scale=1, rotate=0]

% photon going through cycloid cycle in colors
 \foreach \x in {0, 0.05,...,1}
  {
  \definecolor{reddish}{hsb}{\x, 1, 1}
  \draw[draw=none, fill=reddish] 
   ({((2*pi*\x + sin(2*pi*\x r)) },{1 + cos(2*pi*\x r) }) circle [radius=0.5];

  }

 \foreach \x in {0, 0.05,...,1}
  {
  \definecolor{reddish}{hsb}{\x, 1, 1}
  \draw[draw=none, fill=reddish] 
   ({((2*pi + 2*pi*\x + sin(2*pi*\x r)) },{1 + cos(2*pi*\x r) }) circle [radius=0.5];
  }

\path
(2*pi,2) node (topdoublespeed) {}
(3*pi,0) node (deadbottom2) {};
\draw[<-, thick] (topdoublespeed) +(0.5, 0.5) -- ++(2,2) node (topdoublespeedtext){}; 
\draw[<-, thick] (deadbottom2)+(0.4,-0.4) -- ++(2,-2) node (deadbottomtext) {}; 
\path
(topdoublespeedtext) node [anchor=south west,  align=right]
{\textbf{TopDoubleSpeed} \\
Red indicates the highest speed.\\
The maximum speed is double the average. \\
};
\path
(deadbottomtext) node [below , align=right]
{\textbf{DeadBottom} \\
Turquoise blue indicates the slowest speed.\\
The direction of movement is reversed. \\  
There is a slight moment of zero speed. \\
};


\end{tikzpicture}\caption{A single photon sub-particle, ornac, going through cycloid cycle in colors
\label{fig:photon_cycloid_colors}}
\end{figure}

