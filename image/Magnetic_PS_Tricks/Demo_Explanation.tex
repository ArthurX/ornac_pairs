% source = http://ctan.sharelatex.com/tex-archive/graphics/pstricks/contrib/pst-electricfield/pst-electricfield-docEN.tex

\documentclass[11pt,english,english,BCOR10mm,DIV12,bibliography=totoc,parskip=false,smallheadings
    headexclude,footexclude,oneside]{pst-doc}
\usepackage[latin1]{inputenc}
\usepackage{pst-electricfield}
\let\pstEFfv\fileversion

\usepackage{pst-func}
\usepackage{pst-exa}% only when running pst2pdf
%\let\PSTexample\LTXexample        % when not running pst2pdf
%\let\endPSTexample\endLTXexample  % when not running pst2pdf
\usepackage{esint}


\lstset{pos=t,language=PSTricks,
    morekeywords={psElectricfield,psEquipotential},basicstyle=\footnotesize\ttfamily}
\newcommand\Cadre[1]{\psframebox[fillstyle=solid,fillcolor=black,linestyle=none,framesep=0]{#1}}
%
\begin{document}




\section{Method based on electric flux (Patrice Mégret)}

Equipotential surfaces and E-field lines can be drawn by using the package \LPack{pst-func} and the command \Lcs{psplotImp}\verb+[options](x1,y1)(x2,y2)+. 

The following explanations describe the theory on which this is based.


Gauss theorem states that the electric flux across a closed surface $S$ and defined by:
\begin{equation}\label{pm-eq-a}
\psi = \oiint\limits_S \vec{D} \cdot \vec{u}_n \mathrm{d} S = Q
\end{equation} 
is equal to the real charge $Q$ inside $S$. As a consequence, in place where there is no charge ($Q=0$), the electric flux is a conservative quantity.


A tube of flux is a tube constructed on D-field lines and without charge, the flux going inside any cross-section of the tube is equal to the flux going outside any cross-section of the tube. This means that, by following a tube of a given flux, we automatically follow a D-field line. By using this technique, it is thus possible to obtain a scalar equation that describes the D-field lines. This equation is an implicit equation and can be derived for systems with simple geometrical properties.

Here the analysis will be limited to point charges and the D-field lines will thus be identical to the E-field lines as there is no electric polarization.
 

For a point charge $q$, located at the origin of the coordinate system, the electric field and the potential are given by:
\begin{equation}\label{pm-eq-b}
\vec{E} = \frac{1}{4 \pi \varepsilon_0 \varepsilon_r} q \frac{\vec{r}}{|\vec{r}|^3}
\end{equation}
\begin{equation}\label{pm-eq-c}
V = \frac{1}{4 \pi \varepsilon_0 \varepsilon_r} \frac{q}{r} 
\end{equation}

The flux across a portion of a sphere of surface $S$ and with an aperture angle  $\theta$, is simply given by:
\begin{equation}\label{pm-eq-d}
\psi = \varepsilon_0 \varepsilon_r E S = \frac{1}{2} q (1 -\cos\theta) 
\end{equation} 
because $S= 2\pi r^2 (1 - \cos\theta)$ and from (\ref{pm-eq-a}) $4 \pi r^2 \varepsilon_0 \varepsilon_r E =q$.

\begin{center}
\begin{pspicture}(-3,-3)(3,3)
%\psgrid
\psdot[dotscale=2](0,0)
\uput[-135](0,0){$q$}
\psaxes[labels=none,ticks=none]{->}(0,0)(-2.5,-2.5)(2.5,2.5)[$x$,-90][$y$,0]
\pswedge(0,0){2}{-30}{30}
\psarc{->}(0,0){1}{0}{30}
\rput(1.2,0.2){$\theta$}
\rput(2.2,0.7){$S$}
\end{pspicture}
\end{center}

To find the implicit expression of the E-field lines, it is sufficient to express the flux invariance:
\begin{equation}\label{pm-eq-e}
\psi(x,y) = \frac{1}{2} q (1 -\cos\theta) = \mathrm{cte}
\end{equation}
This relation simply shows that E-field lines correspond to $\theta=\mathrm{cte}$, so that they are clearly radial lines.
 
For the E-field lines in the $xy$ plane, expression (\ref{pm-eq-e}) in Cartesian coordinates is:
\begin{equation}\label{pm-eq-f}
\frac{x}{\sqrt{x^2+y^2}} = \mathrm{cte}
\end{equation}

For the equipotential surface, relation (\ref{pm-eq-c}) is already in implicit form, therefor $V=\mathrm{cte}$ is the wanted equation:
\begin{equation}\label{pm-eq-g}
\frac{1}{\sqrt{x^2+y^2}} = \mathrm{cte}
\end{equation}

The following graph shows the field and equipotential for this point charge obtained by implicit plotting of functions (\ref{pm-eq-f}) and (\ref{pm-eq-g}). It is clear that the E-field lines are radial ones and the equipotential surfaces cross the $xy$ plane along circles orthogonal to the E-field lines. 
\begin{center}
\begin{pspicture*}(-5,-5)(5,5)
\psframe*[linecolor=green!20](-5,-5)(5,5)
\psgrid[subgriddiv=0,gridcolor=lightgray,griddots=10]
% \psElectricfield[Q={[1 0 0]}]
% \psEquipotential[Q={[1 0 0]}](-5,-5)(5,5)
\multido{\r=-1+0.1}{20}{%
\psplotImp[linestyle=solid,linecolor=blue](-6,-6)(6,6){%
x y 2 exp x 2 exp add sqrt div \r \space sub}}
\multido{\r=0.0+0.1}{10}{%
\psplotImp[linestyle=solid,linecolor=red](-6,-6)(6,6){%
x 2 exp y 2 exp add sqrt 1 exch div \r \space sub}}
\end{pspicture*}
\end{center}


\begin{verbatim}
%% E-field lines
\multido{\r=-1+0.1}{20}{%
\psplotImp[linestyle=solid,linecolor=blue](-6,-6)(6,6){%
x y 2 exp x 2 exp add sqrt div \r \space sub}}

%% equipotential
\multido{\r=0.0+0.1}{10}{%
\psplotImp[linestyle=solid,linecolor=red](-6,-6)(6,6){%
x 2 exp y 2 exp add sqrt 1 exch div \r \space sub}}
\end{verbatim}



Let's now generalize to point charges distributed arbitrarily along a \textbf{line}.  The charge $i$ is $q_i$ and is placed at $(x_i,0)$. 
\begin{center}
\begin{pspicture}(0,-3)(12,3)
%\psgrid
\psset{dotscale=2}
\dotnode(0,0){NA}\nput{-45}{NA}{$q_1$}
\dotnode(2,0){NB}\nput{-90}{NB}{$q_2$}
\dotnode(5,0){NC}\nput{-90}{NC}{$q_n$}
\dotnode[linecolor=red](4,2){ND}\nput{90}{ND}{$P(x,y)$}
\ncline{NA}{ND}\naput{$r_1,\theta_1$}
\ncline{NB}{ND}\nbput{$r_2,\theta_2$}
\ncline{NC}{ND}\nbput{$r_n,\theta_n$}
\psaxes[labels=none,ticks=none]{->}(0,0)(0,-2.5)(11,2.5)[$x$,-90][$y$,0]
\psarc{->}(5,0){0.7}{0}{116.5}
\rput(6,0.5){$\theta_n$}
\dotnode[linecolor=blue](4,-2){NE}
\nccurve[ncurv=2,linecolor=green!40!black]{ND}{NE}
\end{pspicture}
\end{center}

This problem possesses a cylindrical symmetry: it is thus sufficient to study the field and the potential in $xy$ half-plane and the complete results are obtained by rotation around the $x$-axis. 

By rotation around $x$-axis, the E-field line in $P$ creates a tube of flux. The flux across any surface including $P(x,y)$ and crossing $x$-axis beyond the last charge (the trace of this surface in  the $xy$ plane is drawn in green)  is obtained from (\ref{pm-eq-d}):
\begin{equation}\label{pm-eq-h}
\psi = \frac{1}{2} \sum_{i=1}^{n} q_i (1 -\cos\theta_i) 
\end{equation} 

E-field lines are easily computed by the condition  $\psi = \mathrm{cte}$, which is expressed as:
\begin{equation}\label{pm-eq-i}
\sum_{i=1}^{n} q_i \frac{x-x_i}{\sqrt{(x-x_i)^2+y^2}} = \mathrm{cte}
\end{equation} 
in Cartesian coordinates.

For the potential, the solution is trivial:
\begin{equation}\label{pm-eq-j}
\sum_{i=1}^{n} \frac{q_i}{\sqrt{(x-x_i)^2+y^2}} = \mathrm{cte}
\end{equation} 

\begin{center}
\begin{pspicture*}(-5,-5)(5,5)
\psframe*[linecolor=green!20](-5,-5)(5,5)
\psgrid[subgriddiv=0,gridcolor=lightgray,griddots=10]
\psElectricfield[Q={[1 -2 0][-1 2 0]}]
\psEquipotential[Q={[1 -2 0][-1 2 0]},Vmin=-2,Vmax=2,stepV=0.25](-5,-5)(5,5)
\multido{\r=-2+0.2}{20}{%
\psplotImp[linestyle=solid,linecolor=red](-6,-6)(6,6){%
x 2 add dup 2 exp y 2 exp add sqrt div 1 mul 
x -2 add dup 2 exp y 2 exp add sqrt div -1 mul add 
\r \space sub}}
\multido{\r=-0.5+0.1}{10}{%
\psplotImp[linestyle=solid,linecolor=blue](-6,-6)(6,6){%
x 2 add 2 exp y 2 exp add sqrt 1 exch div 1 mul 
x -2 add 2 exp y 2 exp add sqrt 1 exch div -1 mul add 
\r \space sub}}
\end{pspicture*}
\end{center}

\begin{verbatim}
%% E-field lines 
\multido{\r=-2+0.2}{20}{%
\psplotImp[linestyle=solid,linecolor=red](-6,-6)(6,6){%
x 2 add dup 2 exp y 2 exp add sqrt div 1 mul 
x -2 add dup 2 exp y 2 exp add sqrt div -1 mul add 
\r \space sub}}
%% equipotential
\multido{\r=-0.5+0.1}{10}{%
\psplotImp[linestyle=solid,linecolor=blue](-6,-6)(6,6){%
x 2 add 2 exp y 2 exp add sqrt 1 exch div 1 mul 
x -2 add 2 exp y 2 exp add sqrt 1 exch div -1 mul add 
\r \space sub}}
\end{verbatim}


The last example corresponds to one charge $+1$ in $(-2,0)$ and one charge $-1$ in $(2,0)$. Here we have superposed the results obtained by implicit functions and those obtained by the direct integration of the equations. 
The superposition is perfect, but the method of implicit function is quite slow. Moreover, this method is limited to problem with cylindrical symmetry.


\section{Examples}

\begin{PSTexample}[pos=t]
\begin{pspicture*}(-6,-6)(6,6)
\psframe*[linecolor=lightgray!50](-6,-6)(6,6)
\psgrid[subgriddiv=0,gridcolor=gray,griddots=10]
\psElectricfield[Q={[-1 -2 2][1 2 2][-1 2 -2][1 -2 -2]},linecolor=red]
\psEquipotential[Q={[-1 -2 2][1 2 2][-1 2 -2][1 -2 -2]},linecolor=blue](-6.1,-6.1)(6.1,6.1)
\psEquipotential[Q={[-1 -2 2][1 2 2][-1 2 -2][1 -2 -2]},linecolor=green,linewidth=2\pslinewidth,Vmax=0,Vmin=0](-6.1,-6.1)(6.1,6.1)
\end{pspicture*}
\end{PSTexample}

\begin{PSTexample}[pos=t]
\begin{pspicture*}(-6,-6)(6,6)
\psframe*[linecolor=lightgray!50](-6,-6)(6,6)
\psgrid[subgriddiv=0,gridcolor=gray,griddots=10]
\psElectricfield[Q={[-1 -2 2 false][1 2 2 false][-1 2 -2 false][1 -2 -2 false]},radius=1.5pt,linecolor=red]
\psEquipotential[Q={[-1 -2 2][1 2 2][-1 2 -2][1 -2 -2]},linecolor=blue](-6,-6)(6,6)
\psEquipotential[Q={[-1 -2 2][1 2 2][-1 2 -2][1 -2 -2]},linecolor=green,linewidth=2\pslinewidth,Vmax=0,Vmin=0](-6.1,-6.1)(6.1,6.1)
\end{pspicture*}
\end{PSTexample}


\begin{PSTexample}[pos=t]
\begin{pspicture*}(-5,-5)(5,5)
\psframe*[linecolor=lightgray!40](-5,-5)(5,5)
\psgrid[subgriddiv=0,gridcolor=lightgray,griddots=10]
\psElectricfield[Q={[-1 -3 1][1 1 -3][-1 2 2]},N=9,linecolor=red,points=1000,posArrow=0.1,Pas=0.015]
\psEquipotential[Q={[-1 -3 1][1 1 -3][-1 2 2]},linecolor=blue](-6,-6)(6,6)
\psEquipotential[Q={[-1 -3 1][1 1 -3][-1 2 2]},linecolor=green,Vmin=-5,Vmax=-5,linewidth=2\pslinewidth](-6,-6)(6,6)
\end{pspicture*}
\end{PSTexample}



\begin{PSTexample}[pos=t,vsep=5mm]
\psset{unit=0.75cm}
\begin{pspicture*}(-5,-5)(5,5)
\psframe*[linecolor=green!20](-5,-5)(5,5)
\psgrid[subgriddiv=0,gridcolor=lightgray,griddots=10]
\psElectricfield[Q={[1 -2 0][-1 2 0]},linecolor=red]
\psEquipotential[Q={[1 -2 0][-1 2 0]},linecolor=blue](-5,-5)(5,5)
\psEquipotential[Q={[1 -2 0][-1 2 0]},linecolor=green,Vmin=0,Vmax=0](-5,-5)(5,5)
\end{pspicture*}
\end{PSTexample}

\begin{PSTexample}[pos=t,vsep=5mm]
\psset{unit=0.75cm}
\begin{pspicture*}(-5,-5)(5,5)
\psframe*[linecolor=green!20](-5,-5)(5,5)
\psgrid[subgriddiv=0,gridcolor=lightgray,griddots=10]
\psElectricfield[Q={[1 -2 0][1 2 0]},linecolor=red,N=15,points=500]
\psEquipotential[Q={[1 -2 0][1 2 0]},linecolor=blue,Vmin=0,Vmax=20,stepV=2](-5,-5)(5,5)
\psEquipotential[Q={[1 -2 0][1 2 0]},linecolor=green,Vmin=9,Vmax=9](-5,-5)(5,5)
\end{pspicture*}
\end{PSTexample}

\begin{PSTexample}[pos=t,vsep=5mm]
\psset{unit=0.75cm}
\begin{pspicture*}(-10,-5)(6,5)
\psframe*[linecolor=lightgray!40](-10,-5)(6,5)
\psgrid[subgriddiv=0,gridcolor=lightgray,griddots=10]
\psElectricfield[Q={[600 -60 0 false][-4 0 0] },N=50,points=500,runit=0.8]
\psEquipotential[Q={[600 -60 0 false][-4 0 0]},linecolor=blue,Vmax=100,Vmin=50,stepV=2](-10,-5)(6,5)
\psframe*(-10,-5)(-9.5,5)
\rput(0,0){\textcolor{white}{\large$-$}}
\multido{\rA=4.75+-0.5}{20}{\rput(-9.75,\rA){\textcolor{white}{\large$+$}}}
\end{pspicture*}
\end{PSTexample}

\begin{PSTexample}[pos=t,vsep=5mm]
\psset{unit=0.75cm}
\begin{pspicture*}(-5,-5)(5,5)
\psframe*[linecolor=green!20](-6,-5)(6,5)
\psgrid[subgriddiv=0,gridcolor=lightgray,griddots=10]
\psElectricfield[Q={[1 -2 -2][1 -2 2][1 2 2][1 2 -2]},linecolor={[HTML]{006633}}]
\psEquipotential[Q={[1 -2 -2][1 -2 2][1 2 2][1 2 -2]},Vmax=15,Vmin=0,stepV=1,linecolor=blue](-6,-6)(6,6)
\end{pspicture*}
\end{PSTexample}

\begin{PSTexample}[pos=t,vsep=5mm]
\psset{unit=0.75cm}
\begin{pspicture*}(-5,-5)(5,5)
\psframe*[linecolor=green!20](-5,-5)(5,5)
\psgrid[subgriddiv=0,gridcolor=lightgray,griddots=10]
\psElectricfield[Q={[1 2 0][1 1 1.732][1 -1 1.732][1 -2 0][1 -1 -1.732][1 1 -1.732]},linecolor=red]
\psEquipotential[Q={[1 2 0][1 1 1.732 12][1 -1 1.732][1 -2 0][1 -1 -1.732][1 1 -1.732]},linecolor=blue,Vmax=50,Vmin=0,stepV=5](-5,-5)(5,5)
\end{pspicture*}
\end{PSTexample}

\begin{PSTexample}[pos=t,vsep=5mm]
\psset{unit=0.75cm}
\begin{pspicture*}(-5,-5)(5,5)
\psframe*[linecolor=green!20](-5,-5)(5,5)
\psgrid[subgriddiv=0,gridcolor=lightgray,griddots=10]
\psElectricfield[Q={[1 2 0][1 1 1.732][1 -1 1.732][1 -2 0][1 -1 -1.732][1 1 -1.732][-1 0 0]},linecolor=red]
\psEquipotential[Q={[1 2 0][1 1 1.732 12][1 -1 1.732][1 -2 0][1 -1 -1.732][1 1 -1.732][-1 0 0]},Vmax=40,Vmin=-10,stepV=5,linecolor=blue](-5,-5)(5,5)
\end{pspicture*}
\end{PSTexample}

\begin{PSTexample}[pos=t,vsep=5mm]
\psset{unit=0.75cm}
\begin{pspicture*}(-6,-5)(6,5)
\psframe*[linecolor=green!20](-6,-5)(6,5)
\psgrid[subgriddiv=0,gridcolor=lightgray,griddots=10]
\psElectricfield[Q={[1 -4 0][1 -2 0 12][1 0 0][1 2 0][1 4 0]},linecolor=red]
\psEquipotential[Q={[1 -4 0][1 -2 0][1 0 0][1 2 0][1 4 0]},linecolor=blue,Vmax=30,Vmin=0,stepV=2](-7,-5)(7,5)
\end{pspicture*}
\end{PSTexample}




\clearpage
\section{List of all optional arguments for \texttt{pst-electricfield}}

\xkvview{family=pst-electricfield,columns={key,type,default}}

\nocite{*}
\bgroup
\raggedright
\bibliographystyle{plain}
\bibliography{pst-electricfield-doc}
\egroup


\printindex







\end{document}
