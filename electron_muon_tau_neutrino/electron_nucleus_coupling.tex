\section{Coupling between electrons and nuclea}

A second problem that I hope to address is the following: if an electron is orbiting a nucleus, we know that that only can happen in certain orbits. The orbit, whether spherical or more complicated of shape, needs to be equal to an integer number of times the wavelength of the electron. The Schrödinger equation and related quantum mechanics makes it possible to calculate the orbits and interactions to great precision. But the big question remains, how does the electron know: "I have made one whole loop around the nucleus and now I am on the exact starting point again to make a second loop." 

What is the locking mechanism? The electron is moving through a virtual empty space, far away from the nucleus and somehow it knows or feels that it is no good to make one loop around the nucleus in one-and-a-half wavelength, just unthinkable.

Most introductions into quantum mechanics start with a one  dimensional quantum well. Here the mechanism is clear: we have the side walls of the well, which provide a physical boundary. If we study the sound produced by a guitar by listening only, we can conclude that the tones produced are quantized into twelve semi-tones. This is the correct answer, but not the complete physical explanation. You can not explain the tone forming of a guitar without talking about the frets, which limit the free part of the vibrating string. And for a flute or saxophone, you need to talk about the holes in the side, which define the length of the vibrating air. Otherwise you have discovered the phenomena and can use them in calculations and predictions, but there is more to discover.

\section{De Broglie wavelength and cycloid rotation}

De Broglie was the first one to bring the notion of wavelength back to particles, after Einstein had successfully connected the wavelength of light to the discrete photons. In its most simple form it states, that as the speed of a particle increases, its wavelength becomes smaller. How can the model of matter consisting of sticks of quarks be connected to wavelength? The stick of ornac pairs has a hyper-c field, which is angular confined and extends far, this provides an orientation in 3D space and a connecting force. As the particle rotates, while it is moving forward, the protruding hyper-c field will fix a wavelength, which can connect to other matter, at the moment that the hyper-c fields intersect. It is best compared to the rotating beam of a lighthouse: suppose two lighthouses are far apart, but it can happen that both beams meet and for a short period of time overlap, at that moment the hyper-c force will create its action.
Interesting would be to have a mechanism, which would consistently accelerate the rotation of particles at the moment, in which they gain forward speed.

\section{duplicate? Electron to proton neutron coupling}

The electron consisting of three ornac-pairs as described above and in that way can couple via its hyper-c field to the ornac axis of the nucleus. The coupling is not as strong as a coupling between proton and neutron, because the ornac-pairs of the electron rotate with a much lower rotational speed.

Why is the force between the electron ornac-pairs and those of the nucleus repulsive? The fastest rotating ornac-pairs of the electron are the negative ones with a charge of $\frac{-2}{3}$, these fastest electron ornac-pairs couple best to the slowest rotating ornac-pairs of the nucleus, because then the angular speed difference is lowest and the hyper-c coupling the strongest, so they couple most to the slower moving $\frac{-1}{3}$ ornac-pairs of the nucleus.




