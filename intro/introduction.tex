\chapter{Introduction}


\section{Coupling between light and matter}
Before we go any further, we need to take a new look at the interaction between light and matter. How does mass influence light? When does it couple? 
\paragraph{}
\subparagraph{}
First example, if light travels from one star to another one, where does it move fastest? In the middle, when it is most far away from all masses, it will travel with the highest speed. 

\subsection{Shapiro time delay}
Second example: a satellite circles the sun and is opposite the earth, so the radio waves have to pass along the sun, the radio waves, which are also photons, take longer to reach the earth. So again the mass influences the speed of the photons, it takes longer to travel the same distance. In both cases the photons couple to the masses of the protons and neutrons and move slower. 

Shapiro:
%\footnote{https://en.wikipedia.org/wiki/Shapiro_delay}: 
\begin{quotation}

"Because, according to the general theory, the speed of a light wave depends on the strength of the gravitational potential along its path, these time delays should thereby be increased by almost $2.10^{-4}s$ , when the radar pulses pass near the sun. Such a change, equivalent to 60 $km$ in distance, could now be measured over the required path length to within about 5 to 10\% with presently obtainable equipment."

 \end{quotation}


\subparagraph{}
Third example: if light is send through a long optical fiber, which is laid out in a loop, somewhere on the earth, the rotation of the earth can be measured. The light, which travels along with the rotation of the earth, is a little bit shifted in color. While the light, which travels in the counter direction, is shifted in the opposite way. So what happens is, first the photons couple to the the mass of the earth and so are slowed down, but that influences light moving in both directions equally. So how can the light know it is rotating with respect to the rest of the universe. The photos couple to the rest of all matter around, not only the earth, which is closest by, but also to all other matter.
\subparagraph{}
A fourth example: light diverging, when it passes through a single slit. The photons pass through the slit and couple to the protons and neutrons in the material besides the slit. The coupling exerts a force and bends the photon. Can this be the same coupling mechanism as the light, which is bended when passing along a heavy celestial object?
In this approach we change perspective instead of the Einsteinian space time fabric, we keep time and space constant on a grid and assume a coupling between photons and matter like protons and neutrons. At the same time this coupling influences the speed of the photons and it can bend their path.
An essential part is that the photons move slower without losing energy, just like photons moving through glass. In glass light moves slower, but if it gets out of the glass, we do not assume that it gains an energy boost.


