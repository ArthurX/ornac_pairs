\chapter{Photons Short}

\paragraph{
A short introduction} of the model for photons without details or applications, so that we can continue on to apply the same model to quarks and build nuclea. Which is much more essential. Later on we continue with the photon and how it interacts in more detail.

\paragraph{
A long time ago} I was trying to to make a model of a photon, that somehow would be more easy to imagine and visualize,  and which would allow for the possibility to interact with other particles in a more defined manner. This had to be reached by being more explicit about the photon.

Much advance in physics has been made by assuming and discovering a smaller level and every time one goes from one level to a smaller one, more things can be explained in better details. The atoms which make up the molecules. The electrons and protons, which give rise to a charged particle, if there is a surplus of one kind. In general one can state that even a neutral particle can have charges as long as they are equal.




A photon must generate a field according to Maxwell, which has two components, an electrical and a magnetic field, which are both perpendicular to each other and are perpendicular to the direction in which the photon moves. Both the electrical and the magnetic field oscillate in a sinusoidal manner.

\paragraph{
So my thinking was:} can a photon be replaced by two massless point charges, which have opposite charge and rotate around the central imaginary point in the middle? The central point in the middle goes forward with the speed of light c, and the massless point charges rotate around this moving central point with a frequency that corresponds with their wavelength c.q. energy.
So it is an electrical dipole rotating around the central point, which moves with c.
The question is what will a stationary, non rotating, observer see from these rotating dipoles. As the speed of the whole photon is c, one of the massless point charges \footnote{to be called ornac hereafter} will be moving with a speed above c and the other one with a speed under c.
Here we need to expand electromagnetism to include massless charges moving above the speed of light c. And this only in the exclusive case of rotation.

 A global inspection of the Maxwell equations (with one half closed eye) makes one guess, that the speed of light is the limit, the absolute horizon for interaction between two particles.

And we have to notice that as our small point charges are subparticles of massless photons, our point charges are even more massless. 

\paragraph{
First} we take a look at the electric field. Let us assume that the part of the electric field, which is moving with a speed above c, has to be ignored.
In the case of the rotating dipole one charged particle will be moving above c and thus be invisible and the other one will be moving under c and be visible. The border between visible and invisible part of the electric field lays exactly in the middle at the central point which moves with c.
We now need to further define the movement of the massless dipole. If you follow the movement of a bicycle wheel over a road and you concentrate on one point fixed to the rim, for example the valve, you will notice that the valve moves with twice the speed of the bike when it is at the top of the wheel and that its speed is zero when it touches the ground (otherwise the wheel would slip). This bike wheel movement is beneficial to our model of the photon, which we want to build, because it lets one of the massless charges, stop its motion with regard to the non moving world. This short stop in motion allows for time to interact with particles that do not move. And this short stop in motion coincides with maximum field strength.

\paragraph{
One  can ask:} why do those two particles stay together? Why do they stay connected? (Of course, we know: the static electric Coulomb force.) But it is better to move into the perspective of the photon and position oneself on top of one of the massless charged particles. If you look to the center, you see at the other side another particle, which is stationary, while the rest of the world is moving with variable speeds some parts move above c and hence are invisible. So the solid rock in your existence is that massless particle at the other side. While between the center and your self is an area and stage, where some times appear other charged particles, they move to the center line, slow down, stop for an unendless short time, accelerate and move of the stage. That is the big picture.

So we have a photon, which consists of two charged massless particles, who both rotate around a moving center. And there is a certain area, let's call it the \textbf{sub-c stage}, where interaction with slower moving particles is possible.

\section{Cycloid movement}

We have to dive deeper into the cycloid movement and see what happens where for our use case.

For brevity and easier reference let's give a name to our \emph{"massless charged particle, much less than half a photon"}: \textbf{ornac}.

And as they are always together in a pair of opposing charges, we also have an \textbf{ornac-pair}. Which can be shortened to \textbf{o-pair} with the \textbf{o} indicating the rotating movement.

\subsection{Single sub-particle cycloid}

\begin{figure}[h]\begin{tikzpicture}[scale=1, rotate=0]

% photon going through cycloid cycle in colors
 \foreach \x in {0, 0.05,...,1}
  {
  \definecolor{reddish}{hsb}{\x, 1, 1}
  \draw[draw=none, fill=reddish] 
   ({((2*pi*\x + sin(2*pi*\x r)) },{1 + cos(2*pi*\x r) }) circle [radius=0.5];

  }

 \foreach \x in {0, 0.05,...,1}
  {
  \definecolor{reddish}{hsb}{\x, 1, 1}
  \draw[draw=none, fill=reddish] 
   ({((2*pi + 2*pi*\x + sin(2*pi*\x r)) },{1 + cos(2*pi*\x r) }) circle [radius=0.5];
  }

\path
(2*pi,2) node (topdoublespeed) {}
(3*pi,0) node (deadbottom2) {};
\draw[<-, thick] (topdoublespeed) +(0.5, 0.5) -- ++(2,2) node (topdoublespeedtext){}; 
\draw[<-, thick] (deadbottom2)+(0.4,-0.4) -- ++(2,-2) node (deadbottomtext) {}; 
\path
(topdoublespeedtext) node [anchor=south west,  align=right]
{\textbf{TopDoubleSpeed} \\
Red indicates the highest speed.\\
The maximum speed is double the average. \\
};
\path
(deadbottomtext) node [below , align=right]
{\textbf{DeadBottom} \\
Turquoise blue indicates the slowest speed.\\
The direction of movement is reversed. \\  
There is a slight moment of zero speed. \\
};


\end{tikzpicture}\caption{A single photon sub-particle, ornac, going through cycloid cycle in colors
\label{fig:photon_cycloid_colors}}
\end{figure}



The figure~\vref{fig:photon_cycloid_colors}  shows the idea of cycloid movement for one ornac, a massless charged particle, of a photon. The colors indicate the progress of time. At the top of the arch the color is red and this is the position with the highest speed. The speed is 100\% forward, because the velocity vectors of the rotational movement and the the forward movement point in the same direction. So that makes a forward speed of $2c$.

At the bottom, where the color is turquoise blue, the speed is zero for a very short moment. It is the moment the valve of your bike wheel touches the ground. The velocity vectors point in opposing dirctions and cancel.

\begin{figure}\begin{tikzpicture}[scale=1, rotate=0]

% single ornac of a photon going through cycloid cycle in colors

 \foreach \x in {0, 0.05,...,1}
  {
  \definecolor{reddish}{hsb}{\x, 1, 1}
  \draw[black, fill=reddish] 
   ({((2*pi*\x + sin(2*pi*\x r)) },{1 + cos(2*pi*\x r) }) circle [radius=0.09];
  }
% second cycle
 \foreach \x in {0, 0.05,...,1}
  {
  \definecolor{reddish}{hsb}{\x, 1, 1}
  \draw[black, fill=reddish] 
   ({((2*pi + 2*pi*\x + sin(2*pi*\x r)) },{1 + cos(2*pi*\x r) }) circle [radius=0.09];
  }


% wheel
\path
(0,1) node (hub) {};
\draw[black, thick]
(hub) circle[radius=1];
% arrow

%\draw[->, thick]
%(hub)++(215:1.5) arc  (315:225:1.5);
% spokes
 \foreach \phi in {0, 0.1,...,1}
  {
%  \definecolor{reddish}{hsb}{\x, 1, 1}
  \draw[black, thick] 
   (hub)-- ++(360*\phi :1);
  }

\end{tikzpicture}\caption{Cycloid cycle in colors single ornac of a photon with wheel
\label{fig:cycloid_wheel}}
\end{figure}


Figure~\vref{fig:cycloid_wheel} shows the same, but now with a wheel. And with smaller dots to indicate the position and maybe one can see the speed a little bit better.


\begin{figure}\begin{tikzpicture}[scale=1, rotate=0]

% single ornac of a photon  going through two cycloid cycle in red
% with green oval to indicate the under c area

% greeen oval
\path
(pi,0) node (bottom1){}
(3*pi,0) node (bottom2){};
\draw [draw = none, fill = green]
 (bottom1) circle [ x radius = 0.4, y radius = 1 ]
 (bottom2) circle [ x radius = 0.4, y radius = 1 ]node (greenoval) {};

\draw[<-, thick] (greenoval)+(0.4,-0.4) -- ++(2,-2) node (slowspeed){}; 
\path
(slowspeed) node [below right , align=right]
{Green oval \\ indicates area \\ with  speed below $c$.};


%% red ornac
 \foreach \x in {0, 0.05,...,2}
  {
    \draw[black, fill=red] 
   ({((2*pi*\x + sin(2*pi*\x r)) },{1 + cos(2*pi*\x r) }) circle [radius=0.05];
  }


\end{tikzpicture}\caption{Single ornac of a photon  going through two cycloid cycle in red
 with green oval to indicate the under c area
\label{fig:photon_red_ornac_green_area}}
\end{figure}


Figure~\vref{fig:photon_red_ornac_green_area} shows the same picture, but now with a continiously red charge. The green area indicates the region with a speed below $c$, where the outside world becomes visible to the photon, the sub-c stage. And where the photon becomes visible for the non-rotating world.


\subsection{Duo ornac sub-particle cycloid}
\begin{figure}\begin{tikzpicture}[scale=1, rotate=0]

% two ornac of a photon  going through two cycloid cycle in red

%% red ornac
 \foreach \x in {0, 0.05,...,2}
  {
    \draw[black, fill=red] 
   ({((2*pi*\x + sin(2*pi*\x r)) },{1 + cos(2*pi*\x r) }) circle [radius=0.05];
  }


%% blue ornac
 \foreach \x in {0, 0.05,...,2}
  {
    \draw[black, fill=blue] 
   ({((2*pi*\x + sin((2*pi*\x -pi) r)) },{1 + cos((2*pi*\x -pi) r) }) circle [radius=0.05];
  }

\end{tikzpicture}
\caption{Two ornac of a photon  going through two cycloid cycle in red and blue
\label{fig:two_ornac_cycloid}}
\end{figure}


A photon should consist of two opposing charges. Here both are sketched in their cycloid movement in figure~\vref{fig:two_ornac_cycloid}: blue for a positive charge and red for the negative one.

Thier paths seen over two wavelengths looks quite symmetrical almost like an architectural design with arches dating back to gothic times. It is a bit difficult to visualize, where they will be in reference to each other. But if one is on top, the other is at the bottom. If one is in the middle, the other is also in the middle, but one quater wavelength behind.

The second figure~\vref{fig:two_ornac_cycloid_green_area} adds the green area, where the electric field becomes temporarily visible due to the fact, that the speed is now clearly under $c$. By turn the red and blue charge become visible alternatingly causing an electic field, which modulates like a sinusoid.

An indication of the speeds can one get by looking at the distances between the dots. They are spaced at equal time lapse.



\begin{figure}\begin{tikzpicture}[scale=1, rotate=0]

% two ornac of a photon  going through two cycloid cycle in red and blue
% with green oval to indicate the under c area

% greeen ovals
 \foreach \p in {1,...,3}
    \draw [draw = none, fill = green]
     (pi*\p,0) circle [ x radius = 0.4, y radius = 1 ];

\path
(3*pi,0) node (greenoval) {};
\draw[<-, thick] (greenoval)+(0.4,-0.4) -- ++(2,-2) node (slowspeed){}; 
\path
(slowspeed) node [below right , align=right]
{Green oval \\ indicates area \\ with  speed bellow $c$.};

\path
(1*pi,0) node (greenoval1) {}
(2*pi,0) node (greenoval2) {};
\draw[<-, thick] (greenoval1)+(0.4,-0.4) -- (1.5*pi,-2) node (alternatingly){}; 
\draw[<-, thick] (greenoval2)+(-0.4,-0.4) -- (1.5*pi,-2) node {}; 
\path
(alternatingly) node [below , align=center]
{Positive and negative charges \\ 
come  \textbf{alternatingly} in the \\ 
window of perception.};


%% red ornac
 \foreach \x in {0, 0.05,...,2}
  {
    \draw[black, fill=red] 
   ({((2*pi*\x + sin(2*pi*\x r)) },{1 + cos(2*pi*\x r) }) circle [radius=0.05];
  }


%% blue ornac
 \foreach \x in {0, 0.05,...,2}
  {
    \draw[black, fill=blue] 
   ({((2*pi*\x + sin((2*pi*\x -pi) r)) },{1 + cos((2*pi*\x -pi) r) }) circle [radius=0.05];
  }

\end{tikzpicture}\caption{Two ornac of a photon  going through two cycloid cycle in red and blue
 with green oval to indicate the under c area
\label{fig:two_ornac_cycloid_green_area}}
\end{figure}




\subsection{Photon details of positions}
Here the photon is sketched frozen in one position to illustrate details of the velocity vectors and how they add up.

\begin{figure}\begin{tikzpicture}[scale=1, rotate=0]

% photon vertical with speed vectors

\path 
(0,0) node[circle,draw, fill=red]  (red) {$-\frac{1}{3}$}
(0,6) node[circle,draw, fill=blue] (blue) {$+\frac{1}{3}$};
\draw[black] (red) -- (blue)         node[pos=0.5](center){};
\filldraw
 (center) circle (2pt);
\draw[->, black] (center) -- ++(1,0) node [anchor=west]{speed $c$};
\draw[->, black] (blue) -- ++(2,0) node [anchor=west]{speed $2c$};
\draw[->, dotted] (red) -- ++(0.7,0) node [anchor=west]{speed $\emptyset$};

\draw[->,dashed] (-5,4) to[out=60,in=180] (blue);

\end{tikzpicture}
\caption{Photon vertical with speed vectors 
\label{fig:photon_vertical}}
\end{figure}

\paragraph{
The first position} is the vertical one as figure~\vref{fig:photon_vertical} shows. This one is the most important as here one of the charges, ornacs, has a zero speed and has a visible electric field. While at the same time the other charge, ornac, at the top has a high speed, which creates the magnetic field.


\begin{figure}\begin{tikzpicture}[scale=1, rotate=0]


% photon horizontal with speed vectors



\path 
(0,0) node[circle,draw, fill=red](red) {$-\frac{1}{3}$}
(6,0) node[circle,draw, fill=blue](blue) {$+\frac{1}{3}$};
\draw[black] (red) -- (blue)          node[pos=0.5](center){};
\filldraw
 (center) circle (2pt);
%\draw[->, black] (center) -- ++(1,0) node [anchor=west]{};

%% Arrows indicating the speed components from blue
\draw[->, black, dashed] (blue) -- node[above] { $1c$} ++(2,0) node [](west){};
\draw[->, black, dashed] (blue) -- node[left] { $1c$} ++(0,-2) node [](south){};
\path (west) -- ++(0,-2) node [](southwest){};
%\draw[ black, dotted] (west) -- ++(0,-2) node [](southwest){};
%\draw[ black, dotted] (south) -- (southwest) node []{};
\draw[->, black, thick] (blue) --  (southwest) node[below, right] { $1.4c$};

%% Arrows indicating the speed components from red
\draw[->, black, dashed] (red) -- node[above] { $1c$} ++(2,0) node [](redwest){};
\draw[->, black, dashed] (red) -- node[left] { $1c$} ++(0,2) node [](redsouth){};
\path(redwest) -- ++(0,2) node [](redsouthwest){};
%\draw[ black, dotted] (redwest) -- ++(0,1.5) node []{};
%\draw[ black, dotted] (redsouth) -- ++(1.5,0) node []{};
\draw[->, black, thick] (red) --  (redsouthwest) node[below, right] { $1.4c$};




\end{tikzpicture}\caption{Photon horizontal with speed vectors
\label{fig:photon_horizontal}}
\end{figure}

\paragraph{
In the horizontal position} both ornacs move with a speed above $c$ and are hence invisible.
Figure~\vref{fig:photon_horizontal} shows their vectors.


\paragraph{
In the diagonal} attidude the lower ornac is just under the speed of $c$ and just visible. So the visibility starts at about an eights wavelength before DeadBottom and continues for about a quater of a wavelength. Figure~\vref{fig:photon-diagonal} shows the addition of  their vectors.

To be exacty the speed drops below $c$ at 60 degrees before DeadBottom.
\begin{figure}\begin{tikzpicture}[scale=1, rotate=0]


% photon diagonal with speed vectors



\path 
(0,4) node[circle,draw, fill=red](red) {$-\frac{1}{3}$}
(4,0) node[circle,draw, fill=blue](blue) {$+\frac{1}{3}$};
\draw[black] (red) -- (blue)          node[pos=0.5](center){};
\filldraw
 (center) circle (2pt);
\draw[->,dashed, black] (center) -- node[above] { $1c$} ++(2,0) node [anchor=west]{};

%% Arrows indicating the speed components from blue
\draw[->, black, dashed] (blue) -- node[above] { $1c$} ++(2,0) node [](west){};
\draw[->, black, dashed] (blue) -- node[left] { $1c$} ++(225:2) node [](south){};
\draw[ black, dotted] (west) -- ++(225:2) node [](southwest){};
\draw[ black, dotted] (south) -- (southwest) node []{};
\draw[->, black, thick] (blue) --  (southwest) node (under_c) {};
\path
(under_c)+(0.4,0)node [fill=red!20,draw, rounded corners, right]
{ $0.76c < c \Rightarrow$ blue ornac becomes visible here};

%% Arrows indicating the speed components from red
\draw[->, black, dashed] (red) -- node[above] { $1c$} ++(2,0) node [](redwest){};
\draw[->, black, dashed] (red) -- node[left] { $1c$} ++(45:2) node [](redsouth){};
\draw[ black, dotted] (redwest) -- ++(45:2) node [](redsouthwest){};
\draw[ black, dotted] (redsouth) -- (redsouthwest) node []{};
\draw[->, black, thick] (red) --  (redsouthwest) node [below, right] { $1.84c$};

\end{tikzpicture}\caption{Photon diagonal with speed vectors
\label{fig:photon-diagonal}}
\end{figure}


\paragraph{
The magnetic field} generated in the vertical position is shown in 
figure~\vref{fig:photon_green_yellow}. The ornacs are the blue and red circles on top and the bottom. The blue one on top has doube the speed of light $c$ and generates the magnetic fields. The red one is at DeadBottom and does not move. 

\begin{figure}\begin{tikzpicture}[scale=1, rotate=0]

% photon vertical with green and yellow area for 
% sub and hyper c  indication speed 

\filldraw[fill=yellow, draw=black] (0,0) -- (60:6) 
node[pos=0.75, ](yellow) {} 
arc (60:120:6) -- (0,0);
\draw[->, dashed, black] (yellow) -- ++(2,0) node [right, align=left]
{Yellow area with speed above $c$: \\ 
a special case of the magnetic field: \\
only visible to other particles,  \\ 
that rotate around the same axis, \\
like another photon \\
};

\filldraw[fill=green, draw=black] (0,0) -- (60:3) 
node[pos=0.6, ](green) {} 
arc (60:120:3) -- (0,0);
\draw[->, dashed, black] (green) -- ++(3.4,0) node [right , align=left]
{
Green area with speed below $c$: \\ 
the ordinary magnetic field: \\ 
visible to all};


\path 
(0,0) node[circle,draw, fill=red]  (red) {$-\frac{1}{3}$}
(0,6) node[circle,draw, fill=blue] (blue) {$+\frac{1}{3}$};
\draw[black] (red) -- (blue)         node[pos=0.5](center){};
\filldraw
 (center) circle (2pt);
\draw[->, black] (center) -- ++(1,0) node [anchor=west]{speed $c$};
\draw[->, black] (blue) -- ++(2,0) node [anchor=west]{speed $2c$};
\draw[->, dotted] (red) -- ++(0.7,0) node [anchor=west]{speed $\emptyset$};

%\draw[->,dashed] (-5,4) to[out=60,in=180] (blue);



\end{tikzpicture}\caption{Photon vertical with green and yellow area for \\
 sub and hyper c  indication speed 
\label{fig:photon_green_yellow}}
\end{figure}


\subparagraph{
According to Ampere:} the magnetic dipole moment $m$ caused by a current loop is:


\[ m=I\cdot A  \]

Where $I$ is the current and $A$ is the area.

Let's apply this most simple equation to our rotating ornac-pair and find what the appropriate area is.

The ordinairy magnetic field is generated by that p





\section{Different photon models}

\subsection{Maxwell photon}
The Maxwell model is based on the electro magnetic waves, which propagate with the speed of $c$. It condenses all the current knowledge of that age about electicity and magnetism in one laws. It does not include particles, but certainly is the starting point. And all later photon models should converge to that in the case of a large number of synchronized photons.


\subsection{Einstein photon}
The Einstein photon is the beginning of the quantized phyiscs. It is the explanation of the light induced emmission of electrons. The photon has a defined energy (can it kick an electon out of this material?), which is coupled to the wavelength and the momentum. It is the first evidence that the electro magnetic waves consists of well defined packages. We can not include the work of Max Planck on blackbody radiation as evidence, because there it were assumed to be hypothetical packages only usefull as a mathematical thrick.


\subsection{Dirac photon}
Dirac in his PhD thesis builds on the matrix mechanics of on 


\subsection{Feinman photon}



