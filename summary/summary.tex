\chapter{Summary}

\subparagraph{}
By extending magnetism and defining rules for massless charged particles in a circular movement above the speed of light c, it is possible to replace a photon with two opposing charged particles, that rotate around a central point, which moves with c. This does not violate the Einstein limit on the transfer of information speed, because it is a circular movement.
\subparagraph{}
By using the same extension to magnetism it is possible to replace a quark with two opposing charged particles, of which one is stationary and the other one is circling with a speed above c. The hyper-c magnetism is only visible to particles that circulate around the same axis and in the same direction. So the force field has a narrow angular restriction.
By positioning the quarks along the same axis, it is possible to build stable structures as protons and neutrons, and to connect them seamless together along the same axis. 
\subparagraph{}
Using these structures and this hyper-c magnetic field it is possible to connect and explain a number of different phenomena like: 
gravity,
repulsion of an electron close to a nucleus,
bending of electrons in a double slit experiment,
without using interference,
the same for neutrons, protons and photons,
bending of photons due to gravity and a stable structure for nuclea.
\subparagraph{}
And it is possible to make one prediction:
gravity is directional and this might help explain
a part of the missing dark matter and dark energy.
