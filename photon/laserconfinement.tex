
\section{Laser confinement by photons}

While usually the confinement of neutral atoms by laser beams is explained and depicted by beautiful graphs of the calculated undulating potential, here, I want to paint a different picture. Not pretty, just an empty space in the middle, where the atom can move freely to the left or right and at the sides an area, where there is a chance, that the photons interact with the atom and reverse its direction of motion. The figure below sketches the idea:


\paragraph{}
In his excellent presentation Prof at the be physics conference he treated it in such away, that it invited to think it over again. How and when do the photons couple to the electrons. His whole presentation drives home one point: the quantization of the different speeds of the atoms can only be a multiple of the change of momentum of the photon, when it reverses direction.




\paragraph{}
The yellow dot is the atom moving. And the area between the green lines is the place, where a photon can couple to the atom and reverse the motion. So the atom is locked between the green areas as long as there are enough photons are available to couple. The distance from two green lines tool the next two is half a wavelength.
\paragraph{}
At this moment the green box does not have a well defined size, just let's fix is size to 10\% of the open space. So while moving over the open space, the atom spends 90\% of its time moving freely in the direction of the green box. When it finally arrives at the box, it continues moving forward at its slow speed, but here something changes, the photons start acting. The photons move very quickly, of course, in the time the slow atom moves through the box, the photons pass many open spaces and appear in many boxes. So the photon appears in many different boxes, while the atom stays almost motionless in its single box. In its time in the green box many different photons make their appearance and disappear in the rhythm of the well synchronised laser beam. They have a slightly different position. The size of the atom is much smaller than the wavelength of the photon, so the size of the green box is larger than the atom. The outer radius of the atom is made up by the electrons in the biggest orbit.
\paragraph{}
We know, that the photons, we use for building the laser trap, are detuned. They have a slightly lower energy than the photons, which can be absorbed by the atom and bring it's outer electron in a wider orbit, containing the difference in energy. The photons in the laser trap have a lower energy, they can not kick the electron into its higher orbit, but they can still couple to the outer electron, all be it for the short duration of half a wavelength. In this period of time the photon can make a 180 degree turn and reverse its direction of motion, while at the same moment transferring the same momentum to the atom and also reverse the direction of motion of the atom.

\subsection{Quantized momentum transfer}
How do we know that the photons transfer exactly this amount of momentum? As clearly explained by Prof , suppose,we have a small group of atoms caught in our laser trap. Then we switch off the laser beams and wait a short moment. We make a photo and measure, how the distinct sub groups have moved different distances. Now we can calculate the speeds of the different groups. And so know that they all move with a momentum, which is an integer times the momentum for speed reversal of the photon.
\paragraph{}
The question is, why does the photon make exactly a 180 degree turn and then it is free from the outer electron again? And in what relative position and speed need the photon and outer electron be, to allow this temporary coupling?
\paragraph{}
At this moment we only can speculate. We know that if we increase the number of photons, there is a greater chance for the atoms to reverse direction. We also know that when we lower the energy of the photons, there is a smaller chance that they couple to the atoms.
\paragraph{}
The dimensions are so small, that there is no hope to ever directly observe the mechanism. This does not mean that is useless to make a more detailed model. Even a wrong model will stimulate others to come with a better answer. (Just like Markov's calculation of chances was a direct reaction to…)

\subsection{A more detailed model of a photon}
To make a more interesting model we need to define the photon in more detail. The photon has number of well defined properties: energy, momentum, wavelength, which are connected and measurable and has other properties, which are less defined like the electric and magnetic field,the duration of a single photon ( 10 wavelengths or 4 or half ) and self interference of a single photon.
\paragraph{}
So to make for a more detailed mechanism I propose a model without any pretense. It is just a starting point and hope it will be improved. Let's start. 
I want to reconcile the Maxwellian view of an alternating electric and magnetic wave, with Einstein's view of a particle with a distinct momentum and wavelength and energy (which is all connected into one variable). 
\paragraph{}
A photon has no net charge. So we start with a positive and negative point charge separated by a distance d. The net charge is zero. The average speed of charges is the same as the photon, the speed of light c. While moving forward this constellation rotates around the center point. See the figure below.

The red positive charge at the top moves fast towards the right, while going speedier and speedier down. The blue negative charge starts at zero speed and begins moving up, while slowly gaining speed to the right. After half a wavelength they reversed positions. And the cycle repeats.
This movement is called a cycloid and is best compared to the wheel of a bicycle. The bicycle moves forward with a constant speed, but at the place where the tire of the wheel touches the ground, the speed of the wheel of zero, otherwise it would slip. This kind of movement is called cycloid.
Let's follow the movement of the valve used to inflate the tire.
We start at the ground the valve had zero speed, the wheel continues to rotate, so it moves up accelerating from zero, half way up at the height of the axis, it has the forward speed of the whole bike and the upward does from the rotating wheel, so the direction is 45 degree up and forward. The velocity is the square root of two times the speed of the bike. We go to the top position: the speed of the bike and speed of the wheel are parallel and in the same direction, so the valve has double the speed.
So on average the valve has the speed of the bike, sometimes zero, sometimes double the speed.

We now translate the picture of the bicycle wheel to the rotating photon: we locate the positive charge at the valve and the negative charge at the opposite side of the wheel across the axis. The whole photon moves at the speed of c, so the positive charge sometimes moves with double the speed of c and sometimes it is at a stand still and before it is motionless it moves slowly. And it is this period of slow moving, which enables the photon to interact with slow moving matter like electrons.

For easier reference let's give the photons'  subparticles, the positive and negative charge, a name "ornac". The name is chosen free after Tolkien's Lord of the Rings.
And let's assume they are always created in pairs, so that never a net charge is created. We do not have any clue about the amount of charge they carry, but the smallest known charge is that of the quark, which is  1/3 of the electron charge. Let's assume the ornac pairs come with charges ⅓ or  ⅔ and let's shorten ornac pair to o-pair and let the “o” stand symbol for the  circular or cycloid movement, which they make. Another way to separate the parts of an o-pair, is that one of the o-pairs’ charges is always slow moving sub-c and hence visible and the other one of the pair is fast moving, hyper-c and invisible.

We have to examine this in more detail and add some rules. The first thing is, can anything move faster than the speed of light? It is the Einstein speed limit.
Well, a photon is already special, in that it can move with speed c; a photon is a massless particle, so the subparticles of a photon have even less mass. 
If I have a photon with a structure and it rotates, while moving at speed c, it can not be else than that its parts must sometimes move above c and sometimes below c.

So how do we reconcile this with the Einstein speed limit? Firstly: this only holds for absolutely massless particles.
Secondly: every particle that moves above the speed of light c, is invisible. This means, that the o-pair, which forms the photon, only one o charge is visible at the same moment.
As we are only interested in rotating charges. Thirdly: the only part of the magnetic field, which is visible, is the part which moves under the speed of c. So while our photon rotates and moves forward, we see the fraction of the magnetic field as of the charge was moving over the line representing the speed limit c.

So what do we see of the photon? Alternating we see a positive or negative charge, which comes to a stand still. The slowing charge does not create much of a magnetic field, but the other charge, which then moves with double the speed of c, creates a magnetic field of which half the radius is visible.
As the o-pair moves forward the charges chance roles and hence the magnetic field reverses. This starts to look like a propagating electromagnetic wave.

Let's try to connect to a stationary charge. Suppose it is located one wavelength away. Figure sketches the approach:

Green is the stationary charge

The red charge connects to the green stationary charge and can not move further, altering the movement of the blue charge.



The blue charge swings around the red and green charge. With about double the speed of c, it takes about half the time of a wavelength to finish the semi-circle.

Now the red charge disconnects from the green charge and the photon can continue its path.






Here the path of both the o-pair charges is sketched and one starts to think that the EM fields will cancel each other, but due to the time it takes to complete the semi-circle, which takes half a wavelength delay, they are actually in sync. This is an important result for all forms of reflection, in that it can explain the half wavelength time delay.

After one cycle the first slow o-charge reappears and can connect to the stationary charge. It is now fixed to a point in space and can not continue its cycloid movement. So the fast o-charge continues with the same speed 2c, but instead of its cycloid path it follows now a circle around the fixed point. It takes exactly half a wavelength to complete half the circle.
As we try to rebuild a model of photon interacting with a laser trapped atom, it should get free after 180 degrees, so let this happen. And now we can look at the pattern of the reflection. The top half arcs are the paths of the approaching photon and the dotted lines are the leaving o-pair. You might think that the approaching and leaving path are out of phase, but this sketches their path and the 180 degree swing causes the half wavelength delay in time so they are in the same phase.

This reflection around the green dot is of course not a perfect model, because it does not propose a mechanism for release of the photon from the green dot. We can look back to the atom in the laser trap: firstly the photon does couple to a moving electron

There is this strange thing





Reflection off metal and different stuff

A warning you get when buying a polarizing filter is, that it won't work on metal surfaces, because they don't reflect light, while keeping the polarization. So there are two distinct ways of reflection: one which keeps the polarization, like water, glass etc and one which disturbs the polarization for metals.
Let's start with the metal reflection. We assume that for the size and timescale of a photon, there will be a sea of charges (electrons or their lack of them) and they behave like a super conductor. As a small charge of one of the ornac’s moves slightly above it, this will generate a magnetic field and the sea of charges will generate a magnetic field so that the slow ornac can not enter the super conductor and levitates above it. The fast ornac will swing around the fixated slow ornac. The slow ornac is only fixated in the plane of the metal surface and can freely move over it. So once the fast ornac is swung again out of the metal, the slow ornac is free again.
To show the disturbance of this kind of reflection we need to look at a slanted approach of the photon.

Once the wavelength of the photon becomes too small, it is no longer reasonable to assume a sea of charge by the electrons and the high energy photos pass through in changed.

The reflection of dielectric material.

Here the polarization is kept and the reflection is less sure to happen like in glass, water. Suppose the slow ornac of the o-pair arrives just a little under the surface of the glass air interface, let's say 1/10 of a wavelength. The dielectric material will create a small polarization area in response, this area will stick to the surface, and the show ornac will hang under it. The fast ornac will swing around. In the slanted case it keeps it's orientation.




Directional photon emission in diamond
--------------------------------------------------------------------

The nitrogen vacancies in diamond are used in research…
The photons can only be emitted in four directions. This fact is used so that the sensors only need to catch photons from these known angles. But the question is of course, what is the mechanism. The energy of the future photon is still stored in the orbit of the electron. Figure f shows the idea, but then in 2D. The electron is some where moving along the ellipse and where would it release the photon, so that it can go in its predefined direction. Point A is not logical, neither point B, but one of the points T will do. It is the tangent line to the ellipse and parallel to the direction of the photon. While we can not speculate about the mechanism, why the electron releases the photon there, we know it happens there. Just like a ball which is thrown away, it keeps the direction, which it had just before the release. 

When the elections, which emit the photons, form a part of the covalent bonds in the diamond grid structure, their orbits are locked in space to fixed angles. This only happens in crystals.

Proximity time
--------------------------

For the photon to couple to the electron circling around the nucleus, there needs to be time. The tone to be long enough and close enough to couple.
The visible o part of the photon has a trajectory and the orbiting electron has its own trajectory they have to coincide in time and place for a connection to happen.
Then the visible part of the photon is not fixed to a stationary position, but to the electron in its orbit. The o , which is at roughly double the speed of light c, starts to sling around the electron. As the electron has moved half is orbit and the photon has reversed is direction of movement, they decouple.
That has to be when the photon starts to pull in a direction.


Reflection around a single charge
-----------------------------------------------------





