\chapter{Mesons and exotic hadrons}


\section{Mesons}

Mesons are particles build up of two quarks. The charged ones have a longer lifetime and are thought to be bound by the strong force and the neutral ones are bound by their opposing Coulomb charges. Lets show the picture for the pions with the model of the ornac pairs. Figure  sketches the idea for charged pions. Two ornac pairs are aligned along their axis of rotation and interact via the hyper-c field. Because they both have different angular speeds and radii, they do not see each others full hyper-c field, but only a fraction, if we take the same number as in the proton and neutron example above, it is something like 25\% for the difference between an up and down quark ornac pair.


Mesons are particles made up of two quarks and they decay relative quickly compared to protons and neutrons. In the same family of mesons, there are ones that decay faster than others. The charged mesons are more stable. The neutral ones decay faster. Let's take a look at the pions $\pi$. The charged pion stays alive for $2.6×10^{-8}$ seconds, while the neutral one dies $10^8$ times as quick in $8.4 10^{-17}$ seconds.
What makes the difference? A pion is made up of two up down or its anti-quarks. $ud$ , So in a charged pion there is for example one $\frac{+2}{3}$ ornac-pair and a $\frac{+1}{3}$ ornac-pair, if the pairs are oriented along a single axis and rotate in the same direction, the hyper-c force will be attractive. The hyper-fields will not completely couple as with equal ornac-pairs, while the angular speeds differ, so they only see a quarter of each others field. So the hyper-c field is attractive and the electric force is repulsive, but the electric force is much smaller, so another force like the centripetal force is needed to keep ornac-pairs in balance. The ornac-axis helps as a stabalizing factor.
The neutral pion is build up of an quark and its anti-quark, for example a  $\frac{+2}{3}$ and a  $\frac{-2}{3}$ ornac-pair. If these ornac-pairs would rotate around the same axis in the same direction, the hyper-c force would be repulsive as the charges are opposed. So they are hold together by the electric force, while an orbiting movement provides the centripetal force and if at some moment the ornac-axis cross, then the neutral pion will disintegrate.
The decay modes differ: the charged pion decays into a muon and a muon-neutrino, while the neutral pion falls down into two photons. This should be an excellent moment to calculate the conservation of ornac-momentum.
((So why is this charged pion constellation less stable than a proton?
A proton consists of three ornac-pairs and if we take the top ornac-pair and look at the forces that push and pull it, we see that the pushing force and the pulling force originate from different points, respectively the middle and bottom ornac-pair. The force falls off with $\frac{1}{r^2}$, so only if the opposing forces originate from a different location, there will be a well defined minimum. This minimum gives a proton its stability.
Here in the case of the charged pion, we have two opposing forces, but they both originate from the same location and both fall of with the same relation to distance  $\frac{1}{r^2}$. So the minimum is less defined and is the result of the non-linearities.))


The π± mesons have a mass of $139.6 MeV/c^2$ and a mean lifetime of $2.6\cdot10^{-8} s$. They decay due to the weak interaction. The primary decay mode of a pion, with probability 0.999877, is a purely leptonic decay into an anti-muon and a muon neutrino:

\begin{align*}
\pi^+    &\to    \mu^+    +    \nu \\
\pi^-    &\to    \mu^-    +    \nu 
\end{align*}

Neutral pion decays
The $\pi^0$ meson has a slightly smaller mass of 135.0 MeV/c2 and a much shorter mean lifetime of 8.4×10−17 s. This pion decays in an electromagnetic force process. The main decay mode, with probability 0.98798, is into two photons (two gamma ray photons in this case):

\[ \pi^0 \to 2 \gamma  \]




