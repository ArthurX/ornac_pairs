\chapter{Halo nuclea}



Some isotopes have an halo of one, two or four neutrons, which orbit an otherwise solid nucleus. Examples are helium-6 and -8 with 2 and 4 orbiting neutrons and lithium-11 with 2 orbiting neutrons and C-19 with one neutron. Figure sketches the situation for helium-6: in the middle is an helium-4 nucleus consisting of two deuterium sticks, above and under are positioned the neutrons. The neutrons rotate around their own centerpoint, while circling the helium-4. In figure a the ornac axi of both the neutrons and the ornac axis of the vertical deuterium stick are aligned for a short moment and their hyper-c fields will couple and generate a short impulse, which pulls both the neutrons towards the helium-4 nucleus. As the neutrons rotate further the hyper-c fields no longer overlap, they move in  straight line forward, while rotating, till all the ornac axi are aligned again. It is clear that this can only happen, if all the rotations are well synchronized and the intermittent impulse provides just enough change to the course of the neutrons, that they maintain a stable orbit. This constellation is relative stable with 
a halftime of 800ms. Calculation for Lithium-11
A Binding energy i0.3Mev  is measured. Suppose that this is the speed with which the neutron rotates around the  neutron. Ekineteic  = ½ mv2
mneutron = m = 1.674 ´ 10-27 kg
0.3 MeV = 4.80E-14 Joule
$v^2 =$ 

