\section{Transparent medium}

The movement of a photon through a transparent medium is governed by the electric permittivity of the material. Let's assume the following: the dimension of the atoms of which the material consists is much smaller than the wavelength of the photon. So that a single photon will cause a small displacement of the electrons of a number of atoms. The photon, as build up of the ornac pair moving forward while rotating, has a small and temporary electric field, this induces a small displacement of the electrons of the atoms around it, this small displacement of electrons is a current and will cause a small local magnetic field. This magnetic field will act on the magnetic axis of the photon and turn the photon a little around its axis of forward movement. In the next half of its wavelength, the electric field will change in direction and the induced magnetic field also, so the photon will be turned back a little by the same amount.
As the ornac pairs don't move faster, but they have to cover a longer road, it takes a little bit longer to arrive. 

When the photon exits the material and continues its travel through vacuum, it does not move faster, no energy is added, but it moves in a more straight line. No longer a wobbly line. Is there an advantage in using the ornac pair model? Yes, the trajectory of the ornac moving below the speed of light can be used to accurately model the response of the electrons.


