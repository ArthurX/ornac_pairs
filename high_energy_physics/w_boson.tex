\section{W-boson in neutron decay}

In the decay process of a neutron a $-\frac{1}{3}$ ornac-pair decays into a $+\frac{2}{3}$ ornac-pair. In the process an electron is produced as is necessary to conserve charge and a neutrino is produced to conserve spin. In the character of conservative laws, here ornac angular momentum has to be conserved too.

The electron and the neutrino form together an intermediate particle, the W-boson. The W-boson travels a limited distance and falls apart into an electron and a neutrino. It is possible that the W-boson is a composite particle consisting of the electron and neutrino coupled together in a similar way to a deuterium stick, which is build up of an alternation of positive and negative ornac-pairs, which are held together by their hyper-c field. The great difference is of course in the size of the hyper-c field: for electrons and neutrinos it is much smaller than for nucleides, the force is too weak to hold them together in a stable configuration. The same constellation could hold for the tau and muon and its respective neutrinos.

The neutron consists of three ornac-pairs. The top $-\frac{1}{3}$ ornac-pair moves down below the middle $+\frac{2}{3}$ ornac-pair and there hits the bottom $-\frac{1}{3}$ ornac-pair. They can not annihilate like in a particle anti-particle collision and only emit energy in  the shape of photons. If that would happen, charge would be lost. Instead the ornac-pairs collide and the middle stationary ornac, which provides the charges to the non-rotating world, is kicked out of its stationary position. Conservation of charge in this scenario means, that if an ornac moves out of its stationary center position, an ornac of the same charge must move into a stationary center. The resulting ornac-pairs can each have different angular speeds. The total angular momentum and energy of incoming and outgoing particles will be equal.

Figure X sketches the idea: three ornac pairs are together in a fixed, stable configuration. Then the middle ornac pair wobbles too far. The top and down ornac-pairs hit each other. One of the the stationary ornacs is kicked out of the center. To compensate for the charge and angular momentum, other ornac-pairs are created.




