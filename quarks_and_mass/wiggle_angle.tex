\section{The wiggle angle}

In the model for proton and neutron we have two equal ornac-pairs, who couple fully and attractive, so their ornac-axis are locked and can not easily be turned to another angle, just like two bar magnets that are clicked onto each other. However the ornac-pair in the middle is different: it has an opposing charge. This means that just like in a magnet it is pushing away the top and bottom ones and the axis is not coupled to another ornac-pair. The restraining force, that the middle ornac-pair will not move to the left or right in our picture, has to come from the Coulomb force of the non-rotating charges of the top and bottom ornac-pairs.  The restraining force, that the middle ornac-paid will not tilt in our picture, has to come from the Coulomb force of the rotating charges of the top and bottom ornac-pairs.  While those charges are not completely visible to the middle ornac-pair, due to the difference in angular speed, they still provide the stabilizing force to the middle ornac-pair. The middle ornac-pair hence can wiggle a bit with its axis. And it is this wiggling with its ornac-axis, which makes that its axis scans a wider volume around the narrower axis of the top and bottom ornac-pair and so the middle ornac-pair has a greater chance of coupling to an outside ornac-axis.

\subsection{Limit of the wiggle angle}
If the middle ornac-pair wiggles too much, the neutron or proton gets destabilized. The middle ornac-axis does not align enough with the axis of the top to  bottom and so can not provide the pushing force to keep them apart. The top and bottom ornac-pairs start accelerating to each other and if the middle pair is not quickly realigned, will collide. So a decay mechanism for a neutron would be that the middle ornac-pair wiggles too much, is too late back, the top and bottom ornac-pairs collide and form a new combination of ornac-pairs.

\subsection{The wiggling angle and chance of coupling to an outside particle}
So if we draw a cross section of the sphere surrounding the proton, we have a narrow angle of the top and bottom ornac-pair and a wider angle of the more freely moving middle ornac-pair. If a neutron is in the proximity and rotates, the chances that both of their middle ornac-pairs axis cross and couple is greatest and so the chance on a repulsive coupling is greatest and the neutron will get a little push to start moving away.

While the chance that the middle ornac-pair of one of them will cross with the top to bottom axis of the other is smaller, it will happen and then the coupling is attractive and stronger and so the neuron and the proton will start moving together. The attractive coupling is also stronger, on one hand the ornac-pairs have the same angular speed and couple maximal, in this case four times as strong and and on the other hand this is a coupling between two ornac-pairs and one so the hyper-magnetic coupling is double.

The chance of a coupling is proportional to the solid angle. As the wiggling angle is larger than the hyper-c angle of the top bottom axis, this mechanism of coupling plays a large role. It only plays a role, when there is a certain distance.

\subsection{What interactions does the wiggle angle influence}
The wiggle angle influences a number of interactions : the approach of a neutron to a nucleus and it might trigger decays.


