\section{Simple calculation of hyper-c in proton}

With the ornac-pair model for a proton or neutron as described above, we can try to make a simple calculation of the hyper-c forces involved between the particles. As input we take the radius of the nucleus and its energy content or mass, then we calculate the forces between the ornac-pairs and the contained energy should match. As this is a first and crude approach, we neglect a number of things. We don't care about the energy contained within the quarks, we neglect the electric forces and we accept the uncertainty and model dependence of the radius.
As stated before the force between two magnets or current loops falls off with $  \frac{1}{r^2} $ with $ r $ being the distance between both magnets. This is very convenient as this means the magnetic field is comparable to the gravitational force and electric force at least when the magnets are on a single line and oriented to that line. It also is a conservative field. This formula only holds for distances where $r$ is much larger than the size of the magnet or the current loop.

The proton is build up of two positive ornac-pairs and one negative one sandwiched in the middle, as in the figure. If we ignore the Coulomb force and only look at the hyper-c magnetic field, we see the middle one pushing the top one up away and the bottom one downward. On the other hand the top and bottom one are attracting each other. So the situation is a bit comparable to two springs being compressed by a thick elastic going from top to bottom. Energy is stored as well in the hyper-c field, which is compressed, like in the springs, as in the hyper-c field which holds them together, like the elongated elastic. Both contributions add to the total energy. And if we cut the elastic rope, all the energy stored in both the springs and the elastic rope will be released.
As we assume a stable particle: all forces need to balance, a net force of zero on each sub-particle. The hyper-c forces act only in the direction of the ornac-axis. For reasons of symmetry the middle ornac-pair will experience an equal force pushing it down by the top ornac-pair and an equal force pushing it up by the down ornac-pair, so the net force on the middle ornac-pair is zero. Symmetry also makes the distance between the top and bottom equal.

From the ordinary magnetic field we know that the middle ornac-pair rotates with a lower speed, so its hyper-c field will also have a different angular speed and be only partial visible to the top and bottom ornac-pairs. Let's introduce a coupling factor, the hyper-c visibility factor, $h_y$ , which makes it possible to do a calculation. The force, pushing the top ornac-pair up by the middle pair, has to equal the force, pulling it down to the bottom ornac-pair. As the distance to the the bottom is double the distance to the middle a factor $4$ appears.

$$  h_y \frac{{m_o}^2}{d^2} =   \frac{{m_o}^2}{4 d^2} $$

The hyper-c field between two $+\frac{2}{3}$ ornac-pairs needs to be equal to that between two $-\frac{1}{3}$ ornac-pairs, otherwise gravity and energy between neutrons and protons would never be equal. So their hyper-c magnetic moments must be close to equal and in this formula we have set these equal and squared $m_o$ , the hyper-c magnetic ornac moment.

As the formula shows this is true for all $d$ and for all $m_o$ as long as the hyper-c visibility factor, $h_y$ , is equal to $\frac{1}{4}$. This factor of four follows from the symmetry. During our inspection of sub-c magnetic field of the ornac-pairs we found that the  $-\frac{1}{3}$  ornac-pairs have to rotate with halve the angular speed of the $+\frac{2}{3}$ ornac-pairs. So we can conclude that the coupling hyper-c visibility factor, $h_y$ scales with the square of the angular speed. 

The distance $d$ is fixed by the size of the nucleide. So now we can look up the formula for the energy contained in the hyper-c magnetic field and make the distance and energy match, then we find the hyper-c magnetic ornac moment,  $m_o$, for
The energy of the nuclei is build up of three parts as demonstrated by the springs and elastic rope analogy: the attractive pull between bottom and top ornac-pair, $E_{TopBottom} $ and the two equal repulsive forces, $E_{TopMiddle} $.
$$ E_{Total}=E_{TopBottom} + 2E_{TopMiddle} $$
Each energy component is characterized by a simple function of the distance between ornac-pairs, $r$ and the magnitude of their hyper-c magnetic moment, $m_o$, and their coupling factor, $h_y$. Just like with gravity or electricity.
$$ E(r) = h_y \frac{m_o^2}{r} $$

We can now build up the formula and plugin the numbers. First for the top bottom energy, the distance is $2d$ and the coupling factor is one.

$$ E_{TopBottom} =  \frac{m_o^2}{2d} $$
Secondly for the top middle energy, the distance is $d$ and the coupling factor is $ \frac{1}{4}$.
$$ E_{TopMiddle} =  \frac{1}{4} \frac{m_o^2}{d} $$
Put it all together in the formula above:
$$ E_{Total}= \frac{m_o^2}{2d} + 2 \frac{1}{4} \frac{m_o^2}{d} = \frac{m_o^2}{d}$$
The answer looks too clean to be reasonable, but this all results from the symmetry of the forces and positions.

If we look at the end result for the total energy contained in a particle, it increases as $d$ becomes smaller and is proportional to $\frac{1}{d}$. This is comparable to a photon, where the energy of the photon is related  the same way to its wavelength.




