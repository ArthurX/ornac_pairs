\section{3-Helium to tritium cyclus}

As described before a 3-Helium stick is eager to absorb a neutron at the top end. At the moment it accepts the neutron at the top, it does not form 4-Helium, which is stable, but it releases a proton at the bottom and becomes tritium. It is a bit like the pendulum with four balls, if you let one ball hit the three balls, only the last ball starts moving. 

Tritium is unstable and decays into 3-Helium. How does that look like in the ornac-stick model? The $-\frac{1}{3}$ ornac-pair at the top of the tritium sick has to move to the bottom and change into a  $+\frac{2}{3}$ ornac-pair. Then it will be exactly 3-Helium at the same location. While there are different ways for the top  $-\frac{1}{3}$ ornac-pair to move to the bottom, for example a curved path outside the tritium stick, I have the feeling that a path through the central axis of the tritium stick is viable and necessary in other cases like neutron decay and the 14-N decay as presented below. The $-\frac{1}{3}$ ornac-pair at the top could walk down the tritium stick in a oscillatory movement. By tilting a little at the top, the axis of its hyper-c field would no longer match to that of the whole tritium stick and the repulsive hyper-c field from the second ornac-pair would vanish, the Coulomb force would pull it down inside the positive tritium.

The same cycle is repeated in other nuclea. If we take a look at 14-Nitrogen, we could suppose that it is build up of seven deuterium sticks, but that would make it too similar to 12-C, which is build up of six deuterium sticks and can not absorb a neutron. So 14-N is build up of four deuterium sticks and one tritium and one 3-Helium stick. The 3-Helium stick can absorb a neutron at the top end and releases a proton at the bottom, while forming tritium. So the 14-N becomes 14-C. 14-Carbon decays into 14-N with a half time of 5730 years closing the cycle.
In the process of stellar nucleosynthesis 12-C is build up by the fusion of three 4-Helium nuclea. In analogy 14-C could be build up by the fusion of one 4-Helium and two 5-Helium nuclea, where the 5-He consists of one deuterium and one tritium stick.


