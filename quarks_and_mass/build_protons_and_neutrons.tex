\subsection{Build protons and neutrons}

\subparagraph{First try:} 
let's build proton. Figure shows the idea: two ornac pairs with a stationary charge of +2/3 at the top and bottom and one ornac pair with a stationary charge of -1/3 in the middle. All ornac pairs rotate around the same axis and in the same direction. An observer, who is off axis, only sees the non-rotating parts of the ornac pairs as the total charge of one and the magnetic field of the rotating ornac parts as far as it is under c, which gives rise to the magnetic dipole moment of the proton. At this stage calculations still make no sense, so considerations is the only way to go. Now let's ignore the stationary charges and move to the perspective of the top and bottom ornac pairs: they are both on the same axis and rotate with the same speed and in the same direction, so they both see each others full hyper-c magnetic fields. As they also have the same charge, this force is pulling them together. As the hyper-c field is a magnetic field, it has a force field, which keeps their axi aligned. Just like the ordinary permanent magnets, with which you can play. So we now have the strange situation, that two positively charged quarks or ornac pairs with a stationary positive charge attract each other. Now let's look at the ornac pair in the middle: it has a negative stationary charge and a positive rotating charge, its axis is aligned and it is rotating in the same direction with a different angular speed. The ornac pair in the middle has an opposed charge configuration to the ones on top and bottom, so its magnetic and hyper-c magnetic field is in the opposing direction, it is pushing the top and bottom ornac pairs away, up and down. As the angular speed of the middle ornac pair is different from that of the top and bottom one, their hyper-c fields only couple partially. So the repulsive force of the middle ornac pair towards the top and bottom one is not the same as when their angular speeds would be equal. If the middle ornac pair would have been taken away, the top and bottom ornac pair would crash unbraked together.

\subparagraph{Second try:}
let's build a neutron. A neutron can be built up in the same way. Figure shows the idea. Again all ornac pairs are aligned along the same axis and rotate in the same direction. The top and bottom ornac pairs attract each other through the hyper-c field, which surpasses the coulomb repulsion of their stationary charges. The magnetic character of the hyper-c field connects and stabalizes their axi. In the middle is an ornac pair with a stationary charge of +2/3, so the off-axis observer sees a total charge of zero.

\subsection{Build longer quark-ornac sticks}
\subparagraph{Third try:} 
connect a proton to a neutron to build Deuterium. It is easy to connect a proton and a neutron, if their axi of rotation and their direction of rotation are aligned. Figure shows the idea. The three ornac pairs with a stationary charge of +2/3 connect together and form an angular stabile stick, which is stiff and will not easily bend.  The ornac pairs are connected through the hyper-c field and will keep a straight angle to the axis of rotation. The other three ornac pairs with a stationary charge of -1/3 have a different angular speed and form a different, equally well connected stick. The positively charged stick and the negative stick attract each other following the coulomb force, each one fitting tight between each other. Both ornac sticks can make a small angle with each other. The border between the proton and neutron blurs.

\subparagraph{Fourth try:} 
make an ornac stick of three nuclides and build tritium and helium-3. The same way can be used to connect a third nucleide to the deuterium as shown in figure. Helium-3 and tritium are quite similar and have about the same nuclear magnetic moment. This gives the first hint that sub-c field of the ornac pairs has to be equal for the 2/3 e charged pair and for the 1/3 e one. Helium 3 is a good absorber and detector for neutrons. From the figure it is visible, that a neutron would easily dock at the top or bottom of the helium-3 stick, if the ornac axis of the neutron is aligned to the ornac axis of the helium-3. Another point is that helium-3 does not become helium-4, when it absorbs a neutron. Instead it decays into tritium and a proton. So if a neutron docks to the top of a helium-3 stick, at the bottom the lowwest three ornac pairs detatch as a proton. This is the first pointer that helium-4 should be built in a different way.

\subsection{Connect two quark-ornac sticks}
\subparagraph{
Fifth try:} let's build helium-4. Helium-4 can be built by connecting a deuterium to another deuterium stick. While it might be tempting to align their ornac axi and extend the chain as we have done before, here we have to try a different road and connect them as a cross as shown in figure. If we hold both of them apart and then move the sticks together, we have to overcome the coulomb force as both the particles have a charge of +1. In the end as both sticks are close together, so that their distance is less than the distance between two ornac pairs,  the middle stationary charge of +2/3 e of one stick, will be close to a -1/3 e stationary charge of the other stick and this will pull the two sticks together and form the cross. At the same time the coulomb force, that was first pushing the sticks away, will now be centered in the middle of the cross and no longer push the sticks away from each other, but elongate them. The hyper-c force along their ornac axi stays the same, but the charge that is pushing out has almost doubled. 
Nuclear fusion happens at high temperatures and pressure, besides having a speed the particles are also rotating, so the chances of hitting each other in the middle are small. For lower temperatures the reaction between tritium and deuterium is faster, because tritium has double the number of docking places and those are at a third of the sticks length.

\subparagraph{Sixth try:} 
build lithium-6. If lithium-6 is bombarded with neutrons of sufficient energy it falls apart into tritium and helium-3, so it is built up off two sticks perfectly connected in the middle. Lithium-7 on the other hand is built up out of one tritium and two deuterium sticks. The difference in neutron absorption rate could be explained by the fact, that tritium can not accept a neutron as it already has a neutron on both ends of its stick and and that the deuterium sticks have an open position at one end and are partially shielded by the tritium stick.


