\chapter{Quarks and mass}

The next question is whether the same system can be used for building matter? Ordinary matter consisting of quarks, which themselves are then made up of charged sub-particles rotating partially above the speed of light? The same way we have build up photons out of charged sub-particles, which rotate partially above $c$.    It is certainly possible. 

Quarks are light particles, which for example give mass and charge to protons and neutrons and are strongly bound by a force, conveniently called the strong force. Let's assume that the quark can be build up of one non moving charge, that is fixed in the middle and one opposed charge, which is circling the stationary one. If the circling charge is moving with a speed above c, that charge and the party of the magnetic field above $c$, will be invisible to the rest of the world, which is not rotating around the same axis.

We have to inspect in more detail, the way in which the hyper-c magnetic field works for rotating charges. First let's take a look at a simple situation: we want to know what happens to a test charge or observer, which is not on the axis of rotation of the charged particle pair. Figure shows the idea. Because the observer is off axis, it only sees the non rotating charge in the middle and only that part of the magnetic field, which is under the speed of light c. The second case: we have two charged particle pairs and these pairs both rotate around the same axis, in the same direction and with the same angular speed. Of each pair the rotating charge generates a magnetic field, because they both are perfectly aligned and have the same angular speed, their magnetic fields couple full, as well the part under c as the part above c. If we add an observer, who is located off axis, he will see: both non rotating charges and their magnetic fields under c , but he will be amazed by the large force between them, because the hyper-c magnetic field is beyond his horizon of perception. The third case: the same particle pairs as before, on the same axis, but now one rotating in the other direction. As the rotation is opposing they only see each others normal magnetic field and not the hyper-c field. So while being on the same axis is satisfied, it is also crucial to rotate in the same direction. The fourth case: two particle pairs rotating around the same axis, in the same angular direction, but with a different angular speed.  They both see each others normal magnetic field. The hyper-c magnetic field is more complicated. It is clear that they see each others hyper-c field partially, but not completely as their angular speeds do not match perfectly. Let us assume that coupling between the particle pairs is mutually equal. So if one of the pairs sees 70\% of the others field, it is visa versa.
As we have now clarified the basics of the coupling and limitations of the hyper-c magnetic field, we can use this to build more or less stable matter. The massless charged rotating particle pairs can be connected together with a force, which is much stronger than the magnetic force alone, as long as we make sure that the particle pairs rotate along the same axis and in the same direction. For brevity we introduce the name "ornac pair" to designate the massless charged rotating particle pair, in which one of the charges rotates with a speed above c and hence allows for coupling to the hyper-c magnetic field. Looking to the quark model for matter shows that it makes most sense to assume a charge of +/- 1/3 electron for the charged massless particle and to use some times two of them.

\subparagraph{First try:} 
let's build proton. Figure shows the idea: two ornac pairs with a stationary charge of +2/3 at the top and bottom and one ornac pair with a stationary charge of -1/3 in the middle. All ornac pairs rotate around the same axis and in the same direction. An observer, who is off axis, only sees the non-rotating parts of the ornac pairs as the total charge of one and the magnetic field of the rotating ornac parts as far as it is under c, which gives rise to the magnetic dipole moment of the proton. At this stage calculations still make no sense, so considerations is the only way to go. Now let's ignore the stationary charges and move to the perspective of the top and bottom ornac pairs: they are both on the same axis and rotate with the same speed and in the same direction, so they both see each others full hyper-c magnetic fields. As they also have the same charge, this force is pulling them together. As the hyper-c field is a magnetic field, it has a force field, which keeps their axi aligned. Just like the ordinary permanent magnets, with which you can play. So we now have the strange situation, that two positively charged quarks or ornac pairs with a stationary positive charge attract each other. Now let's look at the ornac pair in the middle: it has a negative stationary charge and a positive rotating charge, its axis is aligned and it is rotating in the same direction with a different angular speed. The ornac pair in the middle has an opposed charge configuration to the ones on top and bottom, so its magnetic and hyper-c magnetic field is in the opposing direction, it is pushing the top and bottom ornac pairs away, up and down. As the angular speed of the middle ornac pair is different from that of the top and bottom one, their hyper-c fields only couple partially. So the repulsive force of the middle ornac pair towards the top and bottom one is not the same as when their angular speeds would be equal. If the middle ornac pair would have been taken away, the top and bottom ornac pair would crash unbraked together.

\subparagraph{
Second try:}
 let's build a neutron. A neutron can be built up in the same way. Figure shows the idea. Again all ornac pairs are aligned along the same axis and rotate in the same direction. The top and bottom ornac pairs attract each other through the hyper-c field, which surpasses the coulomb repulsion of their stationary charges. The magnetic character of the hyper-c field connects and stabalizes their axi. In the middle is an ornac pair with a stationary charge of +2/3, so the off-axis observer sees a total charge of zero.

\subparagraph{
Third try:} connect a proton to a neutron to build Deuterium. It is easy to connect a proton and a neutron, if their axi of rotation and their direction of rotation are aligned. Figure shows the idea. The three ornac pairs with a stationary charge of +2/3 connect together and form an angular stabile stick, which is stiff and will not easily bend.  The ornac pairs are connected through the hyper-c field and will keep a straight angle to the axis of rotation. The other three ornac pairs with a stationary charge of -1/3 have a different angular speed and form a different, equally well connected stick. The positively charged stick and the negative stick attract each other following the coulomb force, each one fitting tight between each other. Both ornac sticks can make a small angle with each other. The border between the proton and neutron blurs.

\subparagraph{
Fourth try:} make an ornac stick of three nuclides and build tritium and helium-3. The same way can be used to connect a third nucleide to the deuterium as shown in figure. Helium-3 and tritium are quite similar and have about the same nuclear magnetic moment. This gives the first hint that sub-c field of the ornac pairs has to be equal for the 2/3 e charged pair and for the 1/3 e one. Helium 3 is a good absorber and detector for neutrons. From the figure it is visible, that a neutron would easily dock at the top or bottom of the helium-3 stick, if the ornac axis of the neutron is aligned to the ornac axis of the helium-3. Another point is that helium-3 does not become helium-4, when it absorbs a neutron. Instead it decays into tritium and a proton. So if a neutron docks to the top of a helium-3 stick, at the bottom the lowwest three ornac pairs detatch as a proton. This is the first pointer that helium-4 should be built in a different way.

\subparagraph{
Fifth try:} let's build helium-4. Helium-4 can be built by connecting a deuterium to another deuterium stick. While it might be tempting to align their ornac axi and extend the chain as we have done before, here we have to try a different road and connect them as a cross as shown in figure. If we hold both of them apart and then move the sticks together, we have to overcome the coulomb force as both the particles have a charge of +1. In the end as both sticks are close together, so that their distance is less than the distance between two ornac pairs,  the middle stationary charge of +2/3 e of one stick, will be close to a -1/3 e stationary charge of the other stick and this will pull the two sticks together and form the cross. At the same time the coulomb force, that was first pushing the sticks away, will now be centered in the middle of the cross and no longer push the sticks away from each other, but elongate them. The hyper-c force along their ornac axi stays the same, but the charge that is pushing out has almost doubled. 
Nuclear fusion happens at high temperatures and pressure, besides having a speed the particles are also rotating, so the chances of hitting each other in the middle are small. For lower temperatures the reaction between tritium and deuterium is faster, because tritium has double the number of docking places and those are at a third of the sticks length.

\subparagraph{
Sixth try:} build lithium-6. If lithium-6 is bombarded with neutrons of sufficient energy it falls apart into tritium and helium-3, so it is built up off two sticks perfectly connected in the middle. Lithium-7 on the other hand is built up out of one tritium and two deuterium sticks. The difference in neutron absorption rate could be explained by the fact, that tritium can not accept a neutron as it already has a neutron on both ends of its stick and and that the deuterium sticks have an open position at one end and are partially shielded by the tritium stick.

\section{
Matching the spin numbers and the nuclear magnetic moments}

At this moment we have a few examples of nuclear structures and we can try to work out, how the nuclear structure connects to the spin numbers and how the different levels of the nuclear magnetic momentum can be build up.
As helium-4 has a zero magnetic field and we suppose that it is built up of two ornac sticks with each three +2/3 ornac pairs and three -1/3 ornac pairs, we have to conclude that the sub-c magnetic field of both kinds of ornac pairs have to be equal to cancel out. The formula for the magnetic moment of a current loop is:
m =  I.A
In which I is the electric current, A is the area and m is the magnetic moment. This can be written as:
m = 1/2 q.r.v
for a charge q, circling with radius r and speed v.

In our case we have two charges and we are interested in the sub-c magnetic field, so the speeds are equal to c. The magnetic moments of the +2/3 ornac pair and the -1/3 ornac pair are equal if the radius of the sub-c magnetic field of the -1/3 ornac pair is double that of the +2/3 ornac pair.
m2/3 = m-1/3
⅔ r2/3 = ⅓ r-1/3
r-1/3 = 2 r2/3

So we know that the angular velocity of the 2/3 ornac pair is double that of the -1/3 pair and that the sub-c surface of -1/3 ornac pair is four times that of the 2/3 pair.
If we first take the one dimensional particles together with an odd number of ornac pairs, like proton, neutron, tritium and helium-3, they all have spin 1/2 and a nuclear magnetic moment of comparable size. If you look at the stick like structure and the rotation in the same direction along the same axis, it makes one think of the simple clean magnetic field of a solenoid. But here the situation is different, because opposing charges are rotating in the same direction. It is more like a couple of permanent bar magnets, with each one hold so that the north poles are directed against each other, with a lot of field lines pushing out between the ornac pairs. 
How can spin 1 look like? Spin 1 has three possibilities: zero magnetic moment, a magnetic moment aligned with the field or one aligned agianst the field. Lithium-6 is the simplest example. It consists of two sticks, with a magnetic moment    : a tritium stick and a helium-3 stick connected in the middle. Without an external magnetic field it forms a cross to evenly divide the charge over the available space. Let's assume that when the external magnetic fiel is switched on, one of the axis of the cross is aligned with the field as shown in the figure. The tritium stick is aligned along    the magnetic field lines and the helium-3 stick is perpendicular to the tritium and to the magnetic field, so the external field will exert a torc force, which aligns the helium-3 with the external field and the tritium. So we get a magnetic moment aligned with the external field. Figure b sketches the situation, when the tritium stick is aligned against the field, the torc again aligns the helium-3 with field and the magnetic moments of the tritium and helium-3 cancel and we get a total magnetic moment of zero.

Some nuclei have spin 3/2, which means they can have a big magnetic moment, a smaller magnetic moment, aligned or anti-aligned, but never zero. Lithium-9 is an unstable isotope, which consists of three tritium sticks. Without an external magnetic field it is supposed to form a 3d cross. Suppose that one of the tritium sticks is already aligned to the external magnetic field. The other two tritium sticks will be perpendicular to the magnetic field and experience a torque force, which will tend to align them as shown in the figure and in that way create a magnet consisting of three aligned tritium sticks. In the other situation one of the tritium sticks is aligned against the magnetic field and will not experience a torque force. The other two tritium sticks will align with the field. So two tritium sticks cancel each others field and only one tritium stick is left over to contribute to the magnetic moment of the nucleus. It should be noted that due to the fact that all the tritium sticks are positively charged, they will push each other away and hence are not all perfectly aligned.

In these simple nuclei it is easy for the sticks to move to other positions unhindered by other sticks which might interfere. It is not always that effortless. For example the decay of widely used in Mosebauer spectroscopy
Deuterium is a bit different story, actually it is the cliff hanger and for a long time has been a show stopper. Deuterium consists of a single stick both ends have have opposing charges rotating in the same direction. So the top has a north pole pointing upward and the bottom has also a north pole pointing downward. In between there are five alternating leaking magnetic fields, which will result in a south pole in the middle. So deuterium can have two orientations to the external filed, which result in a zero torque force and zero magnetic moment and representing spin 0, as shown in the figure. In the first case both the north poles are aligned to the field lines and in the second one the south pole. If deuterium is diagonal aligned there will be a magnetic moment.

As we now have a more clear picture of the magnetic field. It is time to go back to Helium-4. It consists of two deuterium sticks and they both have poles at their ends, two north or two south poles. For helium-4 to exhibit a zero nuclear magnetic moment, it is better if it has two the deuterium sticks which rotate in the same direction so that they both have south poles or north poles. Four equal poles lead to an octopole, which has a low magnetic handle.
Heavier nuclei: C-12 and up
As a star has produced enough helium, helium burning will start.
As we now have the helium-4 disk build up of two deuterium sticks with the same magnetic poles and know how we have to interpret the hints from the quantum spin number, we can try to build Carbon-12 and further. C-12 is build up by fusing three helium disks. They can be aligned along the x, y and z-planes. The deviation of the angle between te two deuterium sticks of 90 degrees within the helium disk will provide some space in the sphere. The helium disks should all have the same magnetic poles. The total magnetic moment will be very evenly distributed along all the sticks and low to the outside world. So C-12 is build up of six ornac sticks protruding out in all the directions of a sphere. To bring the helium disks together one needs to overcome the Coulomb repulsion of the positive charges. While bringing the disks together the ornac sticks elongate and hence store the energy. When the helium disks are centered around the origin, there is no longer a force pushing them away from each other, there is only the positve charge that is pushing all equal charges outward, but now it is balanced around the origin. Some how the positve charged ornac pairs in the center will move closer to the negative ones.

C-13 can be made by adding one neutron to one of the six ornac sticks of C-12. So C-13 consists of five deuterium sticks and one tritum stick. It clearly has one magnet: the tritium stick and spin 1/2. If C-13 absorbs another neutron, it has two tritium sticks and four deuterium sticks. If the tritium sticks are perpendicular to each other, it has two magnets perpendicular and a spin 1. C-14 decays to N-14: at the end of one of the tritium sticks one of the -1/3 ornac pairs decays into a +2/3 ornac pair.

\section{
De Broglie wavelength and cycloid rotation}

De Broglie was the first one to bring the notion of wavelength back to particles, after Einstein had successfully connected the wavelength of light to the discrete photons. In its most simple form it states, that as the speed of a particle increases, its wavelength becomes smaller. How can the model of matter consisting of sticks of quarks be connected to wavelength? The stick of ornac pairs has a hyper-c field, which is angular confined and extends far, this provides an orientation in 3D space and a connecting force. As the particle rotates, while it is moving forward, the protruding hyper-c field will fix a wavelength, which can connect to other matter, at the moment that the hyper-c fields intersect. It is best compared to the rotating beam of a lighthouse: suppose two lighthouses are far apart, but it can happen that both beams meet and for a short period of time overlap, at that moment the hyper-c force will create its action.
Interesting would be to have a mechanism, which would consistently accelerate the rotation of particles at the moment, in which they gain forward speed.







