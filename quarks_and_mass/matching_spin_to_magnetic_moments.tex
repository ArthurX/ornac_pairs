\section{Matching the spin numbers and the nuclear magnetic moments}

At this moment we have a few examples of nuclear structures and we can try to work out, how the nuclear structure connects to the spin numbers and how the different levels of the nuclear magnetic momentum can be build up.
As helium-4 has a zero magnetic field and we suppose that it is built up of two ornac sticks with each three +2/3 ornac pairs and three -1/3 ornac pairs, we have to conclude that the sub-c magnetic field of both kinds of ornac pairs have to be equal to cancel out. The formula for the magnetic moment of a current loop is:
$m =  I.A$
In which I is the electric current, A is the area and m is the magnetic moment. This can be written as:
$m = 1/2 q.r.v$
for a charge q, circling with radius r and speed v.

In our case we have two charges and we are interested in the sub-c magnetic field, so the speeds are equal to c. The magnetic moments of the +2/3 ornac pair and the -1/3 ornac pair are equal if the radius of the sub-c magnetic field of the -1/3 ornac pair is double that of the +2/3 ornac pair.
m2/3 = m-1/3
⅔ r2/3 = ⅓ r-1/3
r-1/3 = 2 r2/3

So we know that the angular velocity of the 2/3 ornac pair is double that of the -1/3 pair and that the sub-c surface of -1/3 ornac pair is four times that of the 2/3 pair.
If we first take the one dimensional particles together with an odd number of ornac pairs, like proton, neutron, tritium and helium-3, they all have spin 1/2 and a nuclear magnetic moment of comparable size. If you look at the stick like structure and the rotation in the same direction along the same axis, it makes one think of the simple clean magnetic field of a solenoid. But here the situation is different, because opposing charges are rotating in the same direction. It is more like a couple of permanent bar magnets, with each one hold so that the north poles are directed against each other, with a lot of field lines pushing out between the ornac pairs. 
How can spin 1 look like? Spin 1 has three possibilities: zero magnetic moment, a magnetic moment aligned with the field or one aligned agianst the field. Lithium-6 is the simplest example. It consists of two sticks, with a magnetic moment    : a tritium stick and a helium-3 stick connected in the middle. Without an external magnetic field it forms a cross to evenly divide the charge over the available space. Let's assume that when the external magnetic fiel is switched on, one of the axis of the cross is aligned with the field as shown in the figure. The tritium stick is aligned along    the magnetic field lines and the helium-3 stick is perpendicular to the tritium and to the magnetic field, so the external field will exert a torc force, which aligns the helium-3 with the external field and the tritium. So we get a magnetic moment aligned with the external field. Figure b sketches the situation, when the tritium stick is aligned against the field, the torc again aligns the helium-3 with field and the magnetic moments of the tritium and helium-3 cancel and we get a total magnetic moment of zero.

Some nuclei have spin 3/2, which means they can have a big magnetic moment, a smaller magnetic moment, aligned or anti-aligned, but never zero. Lithium-9 is an unstable isotope, which consists of three tritium sticks. Without an external magnetic field it is supposed to form a 3d cross. Suppose that one of the tritium sticks is already aligned to the external magnetic field. The other two tritium sticks will be perpendicular to the magnetic field and experience a torque force, which will tend to align them as shown in the figure and in that way create a magnet consisting of three aligned tritium sticks. In the other situation one of the tritium sticks is aligned against the magnetic field and will not experience a torque force. The other two tritium sticks will align with the field. So two tritium sticks cancel each others field and only one tritium stick is left over to contribute to the magnetic moment of the nucleus. It should be noted that due to the fact that all the tritium sticks are positively charged, they will push each other away and hence are not all perfectly aligned.

In these simple nuclei it is easy for the sticks to move to other positions unhindered by other sticks which might interfere. It is not always that effortless. For example the decay of widely used in Mosebauer spectroscopy
Deuterium is a bit different story, actually it is the cliff hanger and for a long time has been a show stopper. Deuterium consists of a single stick both ends have have opposing charges rotating in the same direction. So the top has a north pole pointing upward and the bottom has also a north pole pointing downward. In between there are five alternating leaking magnetic fields, which will result in a south pole in the middle. So deuterium can have two orientations to the external filed, which result in a zero torque force and zero magnetic moment and representing spin 0, as shown in the figure. In the first case both the north poles are aligned to the field lines and in the second one the south pole. If deuterium is diagonal aligned there will be a magnetic moment.

As we now have a more clear picture of the magnetic field. It is time to go back to Helium-4. It consists of two deuterium sticks and they both have poles at their ends, two north or two south poles. For helium-4 to exhibit a zero nuclear magnetic moment, it is better if it has two the deuterium sticks which rotate in the same direction so that they both have south poles or north poles. Four equal poles lead to an octopole, which has a low magnetic handle.
Heavier nuclei: C-12 and up
As a star has produced enough helium, helium burning will start.

As we now have the helium-4 disk build up of two deuterium sticks with the same magnetic poles and know how we have to interpret the hints from the quantum spin number, we can try to build Carbon-12 and further. C-12 is build up by fusing three helium disks. They can be aligned along the x, y and z-planes. The deviation of the angle between te two deuterium sticks of 90 degrees within the helium disk will provide some space in the sphere. The helium disks should all have the same magnetic poles. The total magnetic moment will be very evenly distributed along all the sticks and low to the outside world. So C-12 is build up of six ornac sticks protruding out in all the directions of a sphere. To bring the helium disks together one needs to overcome the Coulomb repulsion of the positive charges. While bringing the disks together the ornac sticks elongate and hence store the energy. When the helium disks are centered around the origin, there is no longer a force pushing them away from each other, there is only the positve charge that is pushing all equal charges outward, but now it is balanced around the origin. Some how the positve charged ornac pairs in the center will move closer to the negative ones.

C-13 can be made by adding one neutron to one of the six ornac sticks of C-12. So C-13 consists of five deuterium sticks and one tritum stick. It clearly has one magnet: the tritium stick and spin 1/2. If C-13 absorbs another neutron, it has two tritium sticks and four deuterium sticks. If the tritium sticks are perpendicular to each other, it has two magnets perpendicular and a spin 1. C-14 decays to N-14: at the end of one of the tritium sticks one of the -1/3 ornac pairs decays into a +2/3 ornac pair.


