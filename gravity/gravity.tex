\chapter{Gravity}

\section{part one}

The force of gravity can be build up by the same hyper-c magnetic field as the strong force. If two ornac-pairs are aligned along the same axis and rotate with the same angular speed, they will experience the same hyper-c force only  dependent on the inverse square of the distance $\frac{1}{r^2}$. The fact, that the force of gravity is so much smaller, only comes from the limit of the angle of coupling between the both ornac-pairs. This angle is really small, so only one ornac-axis will couple between two distinct nuclea and if the nuclea are randomly oriented, this will not happen often. So the big difference between the strong force and gravity only has to do with narrow coupling angle of the hyper-c field.

Why is gravity always attractive? 99\% of the observable mass is there in the shape of up and down quarks, which are ornac-pairs. The difference between them lays in the +2/3 or -1/3 charge and their difference in angular speed. This difference in angular speed, causes the different ornac-pairs to couple weaker in this case, four times weaker, to ornac-pairs of the other kind. So while opposing charges would create hyper-c fields, which push each other away, the fact that their angular speeds differ by a factor two, reduces the pushing forces by a factor four. So the net resulting force is pulling it together.

What will happen at the anti-hydrogen gravity experiment at CERN? In this case it are the same ornac-pairs rotating with the same angular speed, but as anti-matter their charges are exactly opposed and the hyper-c field will be fully coupled and pushing away the anti-matter.


\section{part two}

Directional gravity
The ornac-pair model for matter and light has strong force which has a directional limitation, but the hyper-c magnetic field extends unlimited within its angle. So the gravity force between mass particles

The fact, that the gravitational force is created by the hyper-c field of ornac-pairs, means that gravity is a directional force. In ordinary daily life we do not notice this for a number of reasons. First of all we never touch a single atom. Secondly most atoms consist of more than one ornac axis. So only hydrogen with its single axis would be sensitive to this phenomenon and then best in its atomic form. So we have to move into astronomy to get to places where this might show up. In galaxies there are large clouds of hydrogen. Stars are born when these clouds contract enough to start ignition.

The mass of these clouds is considerable, up to 75\% of the barionic mass of galaxies is estimated to be in this shape. So if the gravitational force of the hydrogen atoms has a directional component, and if this direction lays in the plane of the galactial disk, then the gravitational force exerted by the hydrogen clouds on the other objects in the galaxy is much larger. Suppose that the ornac-axis in the hydrogen rotate in the plane of the galaxy and so the hyper-c field swipes only in the galactic plane and never beams upward or downward, because the orientation of the ornac-axis is no longer randomly divided over a sphere, but limited to a narrow band, the gravitational pull increases with an order of magnitude.
In a typical galaxy only about 4\% of the mass is observed as stars (1\%) and clouds (3\%), the other 96\% of matter is inferred by the curves the stars make around the center and the lack of matter to explain the observed curves. This is the origin for the quest to find the missing unobserved, hence dark matter. So if the hydrogen in the clouds is 30 times more oriented in the plane of the galaxy and of course in the right position, it could help explain the curves of the stars.

Further more one should consider the evolution of the gravitational force. First it is concentrated in a plane like in a galaxy or in a line like in a filament. Later on it gradually transforms. In a cloud it can be conceived, that the atoms keep a orientation, because the collisions between the atoms are not frequent, but if a cloud contracts and a star is formed, the hydrogen atoms collide much more frequently and the preferential orientation in the plane is lost. As the hydrogen is fused and burned into helium and heavier elements, the number of ornac-axis is increased and there is no more possibility for an increase of the  gravitational force by orientation. To visualize what happens let's look at a filament and see how the gravity chances when in a certain region star formation take place. A filament has a typical thickness of one light year and is much longer, suppose that the hydrogen atoms are aligned. If in a section stars form, the hydrogen will lose its orientation and burn. The ornac-axis lose their preferential orientation and the whole section will reduce its gravitational pull on the sections beside it to just a few percent. So the very process of star formation changes the directionality of gravity and slowly filaments and galactial clouds lose their preferential orientation and for the out side observer the gravity diminishes, things start to move away from each other. Maybe this evolutionary process might lead to cosmological expansion and be a source of the percieved dark energy.

Gravitational lensing is also influenced by a preferential gravitational orientation of hydrogen clouds inside lensing galaxies. In some galaxies the luminosity of the stars and hence its indicated mass will not agree with the observed bending of light passing through depending on the amount of hydrogen in oriented state. If light passes through a galaxy going through the disk perpendicular, the hydrogen clouds can have a gravitational preferential orientation, which increases the gravity. On the other hand, if the light is traveling parallel to the disk, they will be no noticeable influence of the hydrogen clouds.

As a separate thought: I remember having read sometime, that protons reaching the surface of the moon coming from the sun in 20\% of the case do not experience gravity and bounce off in a straight line. It could be that their orientation is in a plane that does not cross the moon.




